\documentclass[a4paper]{article}

\setlength{\textwidth}{17 cm}
\setlength{\oddsidemargin}{-0.5 cm}

\usepackage{verbatim}
\usepackage[Q=yes]{examplep} % robust verbatim (eg in description keys)
\usepackage{wasysym}
\usepackage{rotating}
\usepackage{threeparttable}
\usepackage{siunitx}

%  \usepackage{amsmath}

\title{RUMD User Manual [Version 3.5]}

\newcommand{\textind}[1]{\mbox{\scriptsize{#1}}}

\newenvironment{example}%
{\bigskip \textbf{\textit{Example}}\\}%
{$\Box$ \newline}

\begin{document}

\author{The RUMD development team: Thomas B. Schr{\o}der, Nicholas P. Bailey,\\
 Trond Ingebrigtsen, Jesper Schmidt Larsen, Claire Lemarchand, \\
Lasse B{\o}hling, Heine Larsen, Lorenzo Costigliola, Ulf R. Pedersen}

\date{\copyright\ May 2019}

%{\copyright\ 2013}

\maketitle

\tableofcontents

\section{Preliminaries and general information}

This document is meant to be a more complete documentation of the functionality
available to a user of RUMD than is available in the tutorial, 
without describing the internal workings in detail. It is assumed that the
reader has studied the tutorial; there is little point in duplicating what is
explained there. The overall aim is that having studied this manual, the user
should have fairly complete knowledge of what RUMD can do, though not how it
does it.

\subsection{Availability and latest version}

The latest version of this manual, and of the RUMD source code, are available 
at the RUMD website \verb|http:rumd.org|. The version of a given installation
of RUMD can be found in Python after importing the rumd module (see below)
via \verb|rumd.GetVersion()|. If using RUMD compiled
with source code checked out from the
subversion repository then the revision number can be found via
\verb|rumd.Get_SVN_Revision()|.

\subsection{RUMD paths and modules}

When RUMD is correctly installed, the relevant paths should be already in
Python's search path and the import statements should work. When compiling from
source without installing, the relevant paths which need to be included in
Python's search path (via \verb|import sys| and \verb|sys.path.append(...)|)
are \verb|<path-to-source>/lib| and
\verb|<path-to-source>/Python|. Note this changed in version 3.1; before that there were three paths that should be included.


From a software point of view, RUMD consists of two Python 
modules\footnote{Three if the \Q{Tools} module is included, which 
provides a Python interface to the analysis tools.}. The first 
is called \verb|rumd|, is implemented in C++ 
(which in turn makes calls to device functions which run on the GPU) and
represents the bulk of the code. The second is called \verb|rumd.Simulation|, 
which is a pure Python module implementing the \verb|Simulation| class.
Both modules need to be imported near the start of a user script, for example 
as follows

\begin{verbatim}
import rumd
from rumd.Simulation import Simulation
\end{verbatim}
After this the \verb|Simulation| class is available without prefix and classes in the \verb|rumd| module can be accessed by prefixing with \verb|rumd.|, for example to create a Lennard-Jones pair potential object:

\begin{verbatim}
pot = rumd.Pot_LJ_12_6(...)
\end{verbatim}
One can also use the \verb|import *| device to import all names from \verb|rumd|
into the current name space, or 
\verb|from rumd import Pot_LJ_12_6, IntegratorNVT|, say, to import some names. 
This can be convenient but it is not considered
good Python practice because of potential name clashes and the idea that it 
should be clear to someone reading the code which module a function or class 
comes from. For Python access to
the analysis tools it is necessary to import the tools module:
\begin{verbatim}
import rumd.Tools
\end{verbatim}


\subsection{Basic  RUMD scripts}


A minimal RUMD script consists of four steps. (1) Create a \verb|Simulation| object from a configuration file. (2) Create a potential object and attach it to the simulation object. (3) Create an integrator object and attach it to the simulation object. (4) Call \verb|Run(nSteps)| on the simulation object. See the tutorial for examples.

\subsection{Acquiring start-configurations}

The program \verb|rumd_init_conf| can be used to create a configuration file
named \verb|start.xyz.gz|. Type
\begin{verbatim}
rumd_init_conf --lattice=fcc --cells=6,6,6
\end{verbatim}
or
\begin{verbatim}
rumd_init_conf -lfcc -c6,6,6
\end{verbatim}
to make a face centered cubic (fcc) lattice with 6$\times$6$\times$6 unit cells.
Type
\begin{verbatim}
rumd_init_conf -c10 -N800,200 -r1.2 -m -o KobAndersen.xyz.gz
\end{verbatim}
to make a configuration for the binary Kob-Andersen system on a (default) simple cubic (sc) lattice.

Below is a list of options for \verb|rumd_init_conf| 

\begin{description}
\item[\Q{-h, --help}] Print information on using the program.
\item[\Q{-q, --quiet}] Hide program output.
\item[\Q{-l, --lattice=STR}] Set lattice type. Default: sc. Allowed values (n="particles in unit cell"):
 \begin{description}
  \item[sc] Simple cubic (n=1).
  \item[rp] Randomly packed (n=1).
  \item[bcc] Body centered cubic (n=2).
  \item[fcc] Face centered cubic (n=4).
  \item[hcp] Hexagonal close packed (n=4).
  \item[hex] Hexagonal layers in xy-planes (n=2).
  \item[dc]  Diamond cubic lattice (n=8).
  \item[NaCl] Rock salt lattice. (n=2x4).
  \item[CsCl] Caesium Chloride lattice (n=2x1).
 \end{description}
\item[\Q{-c, cells=INT, --cells=INT,INT,INT}] Set number of unit cells. Default: 5.
\item[\Q{-N, --num\_par=INT,INT,... }] Reset particle types in lattice. Note: The sum of particles must be $<$ or $=$ lattice sites.
\item[\Q{-u, --mass=NUM,NUM,... }]  Set masses of types. Default: 1.
\item[\Q{-m, --mix\_positions}]  Make positions of particles on lattice sites random.
\item[\Q{-r, --rho=NUM}] Set number density.
\item[\Q{-L,  --length=NUM, --length=NUM,NUM,NUM}] Change length of box vectors (skip resetting when $<$0).
\item[\Q{-d, --minimum\_distance=NUM}] Do not allow distances shorter than distance given.
\item[\Q{-T, --temperature=NUM}] Temperature of random velocity vectors.
\item[\Q{-s, --seed=INT}] Seed for pseudo random numbers.
\item[\Q{-o, --output=FILE}]  *.xyz or *.xyz.gz output file.
\end{description}


In addition, some configurations are supplied with the source code in the directory Conf. 
Configurations for single component liquids are in the sub-directory 
\verb|Conf/SingleComponentLiquid|, for example \verb|LJ_N1000_rho0.80_T0.80.xyz.gz| is a 
configuration with 1000 particles and density 0.80 which has been at least 
partly equilibrated at temperature 0.80 using the Lennard-Jones potential. One 
should not rely on these configurations being properly equilibrated however.
In particular there is no problem using them for a different potential, 
temperature and density than those specified in the filename. The directory
\verb|Conf/Mol| contains configurations for molecular systems, along with the
associated topology files (\verb|.top|).

\subsection{Units} 
Input and output of RUMD is done in reduced units. E.g. temperatures are not given in Kelvin, Celsius or Fahrenheit but in units of $\varepsilon/k_B$. The table below gives conversion from reduced units to two choices of unit systems.
\begin{table}
\begin{center}
\caption{\label{realUnits} Examples of unit systems}
\begin{tabular}{ l  c | c c }
 	& 	& Molar units & Shifted SI units \\
\hline
Length	& $\sigma$					& \AA		& nm  \\
Energy	& $\varepsilon$					& kJ/mol 	& zJ  \\
Mass	& $m$						& $u$		& yg \\
\hline 
Boltzmann's const. & $k_B$				& 0.0083145 kJ/(mol K) & 0.013806 zJ/K  \\
Coulomb's const. & $k_e$ 				& 1389.3 {\AA} kJ/(mol $e^{2}$) & 230.71 nm zJ/e$^2$  \\ 
\hline
Time		& $\sigma\sqrt{m/\varepsilon}$  	& 0.1 ps 	& ps \\
Temperature	& $\varepsilon/k_B$			& 120.27 K	& 72.432 K\\
Velocity	& $\sqrt{\varepsilon/m}$		& km/s 		& km/s \\
Force		& $\varepsilon/\sigma$			& 16.605 pN 	& pN \\
Pressure	& $\varepsilon/\sigma^3$		& 1.6605 GPa 	& GPa \\
Charge		& $\sqrt{\varepsilon\sigma/k_e}$	& 0.026829 e	& 0.065837 e  \\
Mass density	& $m/\sigma^3$				& 1.6605 g/ml   & g/ml \\
Diffusion  & $\sigma\sqrt{\varepsilon/m}$ 		& 0.001 cm$^2$/s & 0.01 cm$^2$/s \\
\hline
\end{tabular} 
\end{center}
\ \\ Values in standard SI units: $u=1.66053892 \times 10^{-27}$ kg; $e = 1.602176565\times10^{-19}$ C; $\textrm{\AA}=10^{-10}$ m; $1\textrm{ mol}=6.02214129\times10^{23}$; $k_B=1.3806488\times10^{-23}$ J/K; $k_e=\frac{1}{4\pi\varepsilon_0}=8.9875517873681764\times10^9$ J m/C$^2$.  SI prefixes: yocto, y=10$^{-24}$; zepto, z=10$^{-21}$; femto, f=10$^{-15}$; pico, p=10$^{-12}$; nano, n=10$^{-9}$; centi, c=10$^{-2}$; kilo, k=10$^{3}$; mega, M=10$^{6}$; giga, G=10$^{9}$. Note the use of non-SI units for densities: g/ml=1000 kg/m$^3$. Values in the table are truncated to 5 significant digits. 
\end{table} 

\subsection{Precision for floating-point arithmetic}

Like many GPU-based simulation codes, most of the RUMD kernels use 
single-precision floating point arithmetic. Code on the host (the CPU) generally
uses double precision; this is important particularly when summing up the 
particle contributions to the total potential or kinetic energy. The effect of
single precision can be seen in the drift of conserved quantities due to 
round-off error. These include the total momentum, which therefore needs to be 
periodically reset to zero, and the total energy in NVE simulations. Even in 
double precision drift will be observed in long runs; the simplest fix is to
use a thermostat (NVT). A potentially
more serious limitation is the convergence of energy
minimization to so-called ``inherent states'', although we have not investigated
how severe this is in practice.

\subsection{Maximum system sizes}

In version 3.0 a new neighbor-list array structure is used. For systems larger than around 10000 particles space is allocated for a number of neighbors estimated from the global density. If this turns out to be insufficient the neighbor-list is re-allocated, increasing the allocated space by a factor of 2. For a simple Lennard-Jones system with a cutoff of 2.5$\sigma$ and a skin of 0.5 $\sigma$ the memory requirements are of order 800 bytes per particle of which 600 are for the neighbor-list (assuming a density of 1 $\sigma^{-3}$). Explicit, though limited, testing on a GTX 780 Ti (with 3072 MB RAM) is consistent with this, allowing simulation of systems up to $N\sim 3.5\times 10^6$. Using a cutoff of 1.5 $\sigma$ (still with skin of 0.5 $\sigma$) requires roughly a factor of two less space, and explicit testing gives a limit  $N\sim 7.3\times 10^6$.  In these tests 1000 time steps were simulated on a Lennard-Jones fcc crystal with temperature 1.5. The autotuner was not used. The autotuner makes copies of the system so requires more memory, although a separate neighbor-list is not created. Using more than one pair potential will have a significant effect because each pair potential object has its own neighbor-list. Molecular systems also use more memory because they require an exclusion list which has the same structure as the neighbor-list and a hard-coded size of 100 excluded neighbors per particle, which is comparable to the neighbor-list size, leading to a size limit of order half that for atomic systems. On GPUs with compute capability less than 3.0, another limit is relevant, namely the number of thread blocks cannot exceed 65535; with 32 particles per block this limit is 2097120 particles.

\subsection{On the use of the word ``block'' in RUMD}

For historical reasons the word ``block'' is used in two different senses in 
RUMD. This should not cause too much confusion, since the contexts are quite
different. Within the CUDA kernels, ``block'' refers to a thread-block in the
usual CUDA sense. More relevant for the user is the sense of dividing the 
simulation run into (say) 100 blocks, such that there is one output file of a
given category for each block.

\subsection{How to cite RUMD}

If you use RUMD in published work, please cite it by giving the RUMD homepage \verb|http://rumd.org|.
A paper describing the code in detail is available in SciPost Physics.  A BibTeX snippet for this paper is:

\begin{verbatim}
@Article{Bailey-SP-2017-RUMD,
author    = {Nicholas Bailey and Trond Ingebrigtsen and Jesper Schmidt Hansen 
             and Arno Veldhorst and Lasse B{\o}hling and Claire Lemarchand and Andreas Olsen 
             and Andreas Bacher and Lorenzo Costigliola and Ulf Pedersen and Heine Larsen 
             and Jeppe Dyre and Thomas Schr{\o}der},
title     = {{RUMD}: A general purpose molecular dynamics package optimized to utilize {GPU} 
             hardware down to a few thousand particles},
journal   = {{SciPost} Physics},
year      = {2017},
volume    = {3},
number    = {6},
pages     = {038},
doi       = {10.21468/scipostphys.3.6.038},
publisher = {Stichting {SciPost}},
timestamp = {2020-07-16},
}
\end{verbatim}

\section{Pair potentials}


An overview of the available 
potentials is given in table~\ref{pot_list} along with the
arguments needed for each pair of types.

\begin{sidewaystable}

  \advance\leftskip-3cm
  \label{pot_list}
  \caption{List of pair potentials implemented in RUMD.}
  
  \begin{tabular}{ | l | c | c | c | c | }
    \hline
    Name of Potential 
    & \textbf{Definition} $v(r_{ij})$ 
    & Parameters/Constants 
    & SetParams 
    & Name in code \\
    \hline
    Lennard--Jones (12,6) 
    & $4\cdot \epsilon_{ij} \left\{ \left ( \frac{\sigma_{ij}}{r_{ij}} \right )^{12} - \left ( \frac{\sigma_{ij}}{r_{ij}} \right )^{6}  \right\} $
    & $\epsilon_{ij}, \sigma_{ij} $ 
    &  $(i, j, \sigma_{ij}, \epsilon_{ij}, R_{cut})$ 
    & \verb=Pot_LJ_12_6= \\
    Lennard--Jones (m,n) 
    & $ \frac{\epsilon_{ij}}{(m - n)} \left\{ n \cdot\left ( \frac{\sigma_{ij}}{r_{ij}} \right )^{m} - m \cdot \left ( \frac{\sigma_{ij}}{r_{ij}} \right )^{n}  \right\} $
    & $\epsilon_{ij}, \sigma_{ij}, m, n $ 
    & $(i, j, \sigma_{ij}, \epsilon_{ij}, R_{cut})$ 
    & \verb=Pot_gLJ_m_n= \\
    Lennard--Jones (12,6) 
    & $ \epsilon_{ij} \left\{ \left ( \frac{\sigma_{ij}}{r_{ij}} \right )^{12} - 2\cdot \left ( \frac{\sigma_{ij}}{r_{ij}} \right )^{6}  \right\} $
    & $\epsilon_{ij}, \sigma_{ij} $ 
    & $(i, j, \sigma_{ij}, \epsilon_{ij}, R_{cut})$ 
    & \verb=Pot_gLJ_m12_n6= \\
    Lennard--Jones Gaussian 
    & $ \epsilon_{ij} \left\{ \left ( \frac{\sigma_{ij}}{r_{ij}} \right )^{12} - 2\left ( \frac{\sigma_{ij}}{r_{ij}} \right )^{6} 
    - \epsilon_0 \cdot \exp\left[- \left ( \frac{r - r_0}{2 \cdot \sigma_0} \right ) ^2 \right] \right\}$ 
    & $\epsilon_{ij}, \sigma_{ij}, \epsilon_0, \sigma_0, r_0$ 
    & $(i, j, \sigma_{ij}, \epsilon_{ij}, R_{cut}, \epsilon_0, \sigma_0, r_0)$ 
    & \verb=Pot_LJ_12_6_Gauss= \\
    Gaussian core 
    & $ \epsilon_{ij} \cdot \exp\left[ \left ( \frac{-r_{ij}}{\sigma_{ij} } \right ) ^2 \right]$
    & $\epsilon_{ij}, \sigma_{ij}$ 
    & $(i, j, \sigma_{ij}, \epsilon_{ij}, R_{cut})$ 
    & \verb=Pot_Gauss= \\
    Buckingham (exp,six) 
    & $ \epsilon_{ij} \left\{ \frac{6}{(\alpha -6)}\exp \left[\alpha \left ( 1-\frac{r_{ij}}{r_m} \right ) \right] - \frac{\alpha}{(\alpha -6)}\left ( \frac{r_m}{r_{ij}} \right )^{6} \right\} $
    & $\epsilon_{ij}, r_m, \alpha $ 
    &  $(i, j, r_m, \epsilon_{ij}, R_{cut})$ 
    & \verb=Pot_Buckingham= \\
    Inverse Power Law, n = 12 
    & $\epsilon_{ij} \left ( \frac{\sigma_{ij}}{r_{ij}} \right ) ^{12}  $
    & $\epsilon_{ij}, \sigma_{ij} $ 
    & $(i, j, \sigma_{ij}, \epsilon_{ij},  R_{cut})$ 
    & \verb=Pot_IPL_12= \\
    Inverse Power Law, n = 18 
    & $\epsilon_{ij} \left ( \frac{\sigma_{ij}}{r_{ij}} \right ) ^{18}  $
    & $\epsilon_{ij}, \sigma_{ij} $ 
    & $(i, j, \sigma_{ij}, \epsilon_{ij},  R_{cut})$ 
    & \verb=Pot_IPL_18= \\
    Inverse Power Law, arbitrary n
    & $\epsilon_{ij} \left ( \frac{\sigma_{ij}}{r_{ij}} \right ) ^{n}  $
    & $\epsilon_{ij}, \sigma_{ij}, n $ 
    & $(i, j, \sigma_{ij}, \epsilon_{ij},  R_{cut}, n)$ 
    & \verb=Pot_IPL_n= \\
    IPL6/LN
    & $\epsilon_{ij} \left( \frac{\sigma_{ij}}{r_{ij}} \right) ^{6}  \ln \left( \frac{r_{ij}}{\sigma_{ij}} \right) ^{-1}   $
    & $\epsilon_{ij}, \sigma_{ij} $ 
    & $(i, j, \sigma_{ij}, \epsilon_{ij},  R_{cut})$ 
    & \verb=Pot_IPL6_LN= \\
    Dzugutov 
    & $ A \cdot \Big( r^{-n} - B \Big) exp\Big( \frac{c}{r_{ij} - a} \Big) +  B \cdot exp\Big( \frac{d}{r_{ij}-b}\Big) $
    & $A, B, a, b, c, d$ 
    & $(i, j, A, B, a, b, c ,d)$ 
    & \verb=Pot_Dzugutov=\\
    Shifted-force Coulomb
    & $\epsilon_{ij}/r_{ij} \textrm{[+force-shifting term]}$
    & $\epsilon_{ij}$
    & $(i, j, \epsilon_{ij}, R_{cut})$
    & \verb=Pot_ShiftedForceCoulomb=\\
    \hline
  \end{tabular}
\end{sidewaystable}



\subsection{Number of types}


Previously there was a hard-coded limit to the maximum number of types, set to 16 by default. In RUMD 3.0 this limit was removed, because parameters for potentials are now stored differently.

\subsection{Cut-offs}


For almost all pair potentials the
constructor takes a required argument \verb|cutoff_method| which is one of
\verb|rumd.ShiftedPotential|, \verb|rumd.ShiftedForce| or \verb|rumd.NoShift|. 
The traditional
choice is shifted-potential, but in our group we use shifted-force cutoffs 
more and more. Note that neither the cut-off method nor the cut-off distance
may be specified for Dzugutov potential, which has an exponential cut-off
 built into the potential itself.

\subsection{Attaching potential objects to the simulation object}

For a simple system with a single pair potential, one typically uses
\verb|rumd.AddPotential(pot)|, where \verb|pot| is the potential object. The old method \verb|SetPotential| has been deprecated in Version 3.5. The difference was that \verb|SetPotential| also cleared any previously added potentials. If that is necessary, one can simply supply the argument \verb|clear=True| to \verb|AddPotential.|
Potentials for intra-molecular forces (bond, 
bond-angles, and dihedral angles) were previously added via special methods but as of Version 3.5 they work like any other potential; one creates an object, calls \verb|SetParams|, and then \verb|AddPotential|. These potentials will be described in Section~\ref{molecules}.


\subsection{General methods for pair potentials}

\begin{description}
\item [Constructor] As mentioned above, almost all pair potentials take a constructor argument specifying the cutoff method. Some also take additional arguments, which describe aspects of the potential which are fixed for all types. An example is the general inverse power law potential \verb|Pot_IPL_n|:

\begin{verbatim}
pot = rumd.Pot_IPL_n(13.6, rumd.ShiftedPotential)
\end{verbatim}
Another example is the generalized Lennard-Jones potential \verb|Pot_gLJ_m_n| which takes two additional arguments, the exponents $m$ and $n$.
\item [SetParams(...)] Set the parameters for a given pair of types (described 
in the tutorial). The exact arguments depend on the pair potential. To find out
what they are within python, type \verb|help(rumd.Pot_LJ_12_6.SetParams)|
(for example). Typically the first two arguments are the two types \verb|i| and
\verb|j| involved in a given interaction. New in version 2.1 is that Newton's 
third law is assumed by default, so only it is no longer necessary to call 
\verb|SetParams| both for types 0 and 1 and then 1 and 0, say.
\item [\Q{SetID_String(pot_str)}] Set the string which identifies a
 potential (in for example the energies file if the potential is being computed
 in addition to the actual simulation potential, or is one term of the latter).
\item [\Q{GetID_String()}] Get the ID string for this potential.
\item [\Q{SetVerbose(v)}] If \verb|v| is false, messages to standard output are
 suppressed. 
\item [SetNbListSkin(skin)] Set the skin for the neighbor list (the width of 
the extra shell of neighbors which allow some time steps to be taken before 
re-building).
\item [GetNbListSkin()] Get the skin for the neighbor list.
\item [\Q{SetAssumeNewton3(bool assume_N3)}] Pass \verb|False| to disable symmetric 
interactions between unlike types by default. Pass \verb|True| to enable (default: 
\verb|True|).
\item [\Q{GetPotentialEnergy()}] Return the potential energy contribution from this
pair potential. This will cause a kernel launch; it does not use the results from
a previous call to \verb|CalcF|.
\item [\Q{ WritePotentials(simBox)}] Write the pair potentials and pair forces 
(for all type-combinations) to a file 
(whose name consists of \verb|potentials_| plus the ID string, with a 
\verb|.dat| extension). A 
simulation box is passed so that the appropriate range of $r$ values is known.
It is more convenient, however, to call \verb|WritePotentials| method in 
\verb|Simulation|, which will call it on all attached pair potentials and 
automatically pass the simulation box.
\end{description}

\section{Other potentials}

\subsection{Coulomb}

The Coulomb potential is on the face of it a pair potential, but requires special consideration because its long range means that generally a straightforward implementation as a pair potential is not appropriate both for performance reasons and due to issues related to periodic boundary conditions. Typically an Ewald-based method is used, such as particle-mesh Ewald (PME). We plan to implement PME but have implemented a direct pair-sum using a shifted force cut-off, which has been shown to give reasonable accuracy and performance in bulk ionic systems with a cutoff of order 5-6 inter-particle spacings [Hansen {\em et al.}, J. Phys. Chem. B, 116, 5378, 2012]. This implementation is the pair potential \verb|Pot_ShiftedForceCoulomb|. The user cannot choose the cutoff method here: Shifted-force is enforced. The cut-off itself is chosen by the user; note that because of the large number of neighbors the memory use is significant.

\subsection{Many-body potential for metals: Effective Medium Theory}

The effective medium theory approach in metallic interactions has been developed by researchers at the Technical University of Denmark. The potential is described in ``A semi-empirical effective medium theory for metals and alloys'', K. W. Jacobsen et al, Surface Science, Vol. 366, pp394-402 (1996). Parameters for six metals are given there using eV as the energy unit and the bohr radius as the length unit. The following snippet shows how to set up a potential for pure gold using the parameters from the 1996 paper but with lengths converted into \si{\angstrom}.

\begin{verbatim}
bohr   = 0.5291772
E0     = -3.80
s0     = 3.00 * bohr
V0     = 2.321
eta2   = 1.674 / bohr
kappa  = 2.873 / bohr
Lambda = 2.182 / bohr
n0     = 0.00703 / (bohr*bohr*bohr)

pot = rumd.EMT_Potential()
# set parameters for particle type 0 to those for Au
pot.SetParams(0, E0, s0, V0, eta2, kappa, Lambda, n0)

\end{verbatim}
Note that alloys can be simulated without any new parameters.



\subsection{TetheredGroup}

This potential allows you you identify a group of particles which are to be
``tethered'' to a set of so-called ``virtual lattice sites'' by harmonic 
springs. The particles are currently identified by type: A list of types is 
passed to the constructor, and all particles of any of those types are considered 
part of the group\footnote{This typically means that extra types have to be 
introduced which are otherwise identical to existing types, just for identifying sets of
particles for tethering. A more general particle selection mechanism will be 
implemented in a future version.}. The spring constant is the same for all 
particles in the group, and the virtual lattice sites can be displaced using
the Move function (typically inside a method to be called using the 
``run-time action'' mechanism (see section~\ref{runtime_actions}); this can be
used to implement constant-speed boundary-driven shearing, for example.

\begin{verbatim}
potWall1 = TetheredGroup(solidAtomTypes=[0], springConstant=500.0)
sim.AddPotential(potWall1)

potWall1.SetDirection(2) # z direction (default is 0 ie x)
potWall1.Move(0.1) # move the virtual lattice sites a common amount
\end{verbatim}

\section{Integrators}

This section provides details on the available integrators listed in the RUMD tutorial. We start with some common methods available on all integrators (listing only those of interest to the user).

\subsection{Common methods for integrators}

\begin{description}
\item [GetTimeStep] Return the current time step
\item [SetTimeStep] Set the time step
\item [GetKineticEnergy] Return the total kinetic energy (i.e. not per particle). In Leap-Frog-type integrators, this requires either calculating a half time step to get the velocities synchronous with the positions, or a full time step to get the ``forward'' velocities, and taking an average of the kinetc energy based on the backward and forward velocities. For the SLLOD integrator this is not required as the kinetic energy is a conserved quantity.
\end{description}



\subsection{IntegratorNVE, IntegratorNVT}

\verb|IntegratorNVE| realizes Newtonian NVE dynamics via the Leap-frog algorithm whereas \verb|IntegratorNVT| realizes 
Nose-Hoover NVT dynamics [S. Nose, J. Chem. Phys. 81, 511 (1984)]. The implementation of 
the Nose-Hoover NVT dynamics is detailed in [S. Toxvaerd, Mol. Phys. 72, 159 (1991)]. In this implementation, 
the Leap-frog algorithm naturally appears by setting the thermostat variable to zero. The former two integrators are thus based on the same 
common C++ class \verb|IntegratorNVT| where \verb|IntegratorNVE| is simply a wrapper for this class with fixed parameters.
Examples of use may be found in the tutorial but are also listed below. Nose-Hoover NVT dynamics can be run with

\begin{verbatim}
# Nose-Hoover NVT dynamics with time step dt = 0.0025 and temperature T = 1.0
itg = rumd.IntegratorNVT(timeStep=0.0025, targetTemperature=1.0, thermostatRelaxationTime=0.2)
\end{verbatim}
The third constructor argument is the relaxation time of the thermostat, which can also be set using

\begin{verbatim}
itg.SetRelaxationTime(0.2)
\end{verbatim}
A linear temperature ramp can also be associated with the NVT dynamics via

\begin{verbatim}
itg.SetTemperatureChangePerTimeStep(0.01)
\end{verbatim}
This causes \verb|targetTemperature| to be changed every time step with $\Delta T = 0.01$ until the simulation is complete. Alternatively, Newtonian NVE dynamics can be run with

\begin{verbatim}
# Newtonian NVE dynamics with time step dt = 0.0025.
itg = rumd.IntegratorNVE(timeStep=0.0025)
\end{verbatim}
Note: No correction of the energy $E$ is performed, and round-off errors will eventually cause the energy to drift over a long simulation. One should always 
estimate whether this drift is significant for the phenomena under investigation.

\subsection{IntegratorNPTAtomic}
\verb|IntegratorNPTAtomic| realizes isothermal-isobaric dynamics via Leap-frog algorithm coupled with a thermostat and a barostat.
In the current version of RUMD, NPT dynamics is not implemented for molecules.
Details on the equations used to implement an isothermal-isobaric ensemble can be found in [Martyna \emph{et al}, J.Chem.Phys. 101, 5, (1994)].
The following is an example of how to use the NPTAtomic integrator.

\begin{verbatim}
# NPT dynamics with time step dt = 0.0025
# T = 2.0, thermostat relaxation time = 0.4, p = 5.0, barostat relaxation time = 10.0
itg=IntegratorNPTAtomic(timeStep=0.0025, targetTemperature=2.0,
thermostatRelaxationTime=0.4, targetPressure=5.0, barostatRelaxationTime=10.0)
\end{verbatim}

Instantaneous volume is not a default output. It is possible to print instantaneous volume in the energy files using \Q{sim.SetOutputMetaData}:

\begin{verbatim}
# print instantaneous volume in the energy files. 
sim.SetOutputMetaData("energies", volume=True)
\end{verbatim}

\subsection{IntegratorNPTLangevin}

This integrator realizes the NPT ensemble using different dynamics. The Langevin equation is used both for particles and for the box dynamics. The method is based on N. Gr{\o}nbech-Jensen and Oded Farago, J. Chem. Phys. 141, 194108 (2014). It has been reformulated as a leap-frog-type algorithm, as outlined in Gr{\o}nbech-Jensen and Farago, Comput. Phys. Commun. 185, 524-527 (2014) (the latter covers the NVT case, but the equations for the box degree of freedom parallel those for the particles). A slight difference from their version is to define the random forces as integrals over the offset-timesteps (from $t_{n-1/2}$ to $t_{n+1/2}$)
instead of taking the mean of two successive non-offset integrals, which is what comes out when straightforwardly reformulating to a leap-frog version.

The user obviously must specify time step, target temperature and target pressure. In addition a friction $\alpha$ must be specified for the particles, and a friction  $\tilde \alpha$ and inertial constant $Q$ must be specified for the box dynamics. The inertial constant is like a mass, although it has different units. See below for advice on the choice of 

By calling SetBarostatMode with arguments rumd.Off, rumd.Iso or rumd.Ansio, the user may disable barostatting (giving thus an NVT ensemble), or isotropic barostatting (default) or anisotropic barostatting. For the latter the box size fluctuates only in one direction, by default the z-direction. The direction may be set by calling \verb|SetBoxFlucCoord|.

\begin{verbatim}
itg = rumd.IntegratorNPTLangevin(timeStep=dt, targetTemperature=T, friction=0.1,
targetPressure=0.1, barostatFriction=1.e-3, barostatMass=1.e-2);
# make the barostatting anisotropic, specifically in the x-direction
itg.SetBarostatMode(rumd.Aniso)
itg.SetBoxFlucCoord(0)
\end{verbatim}

To see the volume fluctuations in the energies files, add the following line:

\begin{verbatim}
sim.SetOutputMetaData(''energies'', volume=True)
\end{verbatim}

There is an important difference between the ensemble generated by this integrator and that generated by IntegratorNPTAtomic. Momentum is not conserved by the dynamics. This means (1) the SetMomentumToZero function, which is called to curb drift in momentum due to round-off error, does nothing; and (2) the ensemble allows non-zero values of the total momentum (although centered around zero due to the friction). In fact this is a more accurate NPT ensemble, whereas the one generated by the other integrators has fixed total momentum so is technically a different ensemble. A consequence is that the number of degrees of freedom when estimating the temperature is 3N, rather than 3N-3, which is otherwise assumed in rumd. This difference is not handled automatically, so the user must include a line in their script to set the number of degrees of freedom correctly:

\begin{verbatim}
sim.sample.SetNumberOfDOFs(sim.GetNumberOfParticles()*3)
\end{verbatim}


\subsubsection{Choosing friction parameters and box inertia}

The user has a certain freedom when choosing the atomic friction parameter. The time scale for a particle velocity to decay is $m/\alpha$ where $m$ is the particle mass. A large value of $\alpha$ gives a fast time scale and strong coupling to the heat bath, and distorts single-particle dynamics substantially, but can help equilibrate the system quickly. For minimal distortion of particle dynamics a low value of $\alpha$ can be chosen in which case longer equilibration times could be required. (If the system has a long relaxation time anyway, e.g. a viscous liquid, then this is not an issue).

Next consider the box inertia $Q$. First note that $Q$ has dimensions mass/length$^4$ or density/length. By linearizing the box equation of motion around the desired pressure (with corresponding volume $V_0$) and omitting the stochastic terms we get an approximate equation for a basic ``breathing'' mode involving volume fluctuations $\Delta V$

\begin{equation}
Q \frac{d^2 (\Delta V)}{dt^2} = \frac{K_s}{V_0} \Delta V
\end{equation}
where $K_s$ is the adiabatic bulk modulus. This has a frequency
\begin{equation}\label{tau_ad}
  \omega_{ad}=\sqrt{K_s/(QV_0)}=2\pi/\tau_{ad}.
\end{equation}
A choice consistent with that made by the authors in their example is that $\tau_{ad}$ should be of the order of the time for sound to travel one inter-particle spacing $a$,

\begin{equation}
  \tau_{ad} = a/c_s = a \sqrt{\rho_m/K_s}
  \end{equation}
where $c_s$ is the sound velocity and $\rho_m$ is the mass density. Combining with Eq.~(\ref{tau_ad}) gives the following equation for $Q$ which does not involve $K_s$:

\begin{equation}
Q = \frac{a^2}{V_0} \frac{\rho_m}{(2\pi)^2}
\end{equation}

For a Lennard-Jones system with density of order 1 and around 1000 particles this gives $Q \sim 3  \times 10^{-5}$, as used by Gr{\o}nbech-Jensen and Farago. Note that the inertial parameter here must be chosen inversely proportional to the system volume.

For box friction parameter $\tilde\alpha$, a very small value could lead to underdamped box dynamics, and thus long lived oscillations of the volume, which are not desireable. A natural choice would be one which implies critical damping of the box dynamics, that is the damping time scale matches that of the adiabatic breathing mode, or more precisely,

\begin{equation}
 \frac{\tilde\alpha}{Q} = 2\omega_{ad} 
\end{equation}

Given the choice of $Q$ above this implies that

\begin{equation} \label{tilde_alpha}
\tilde\alpha = \frac{a}{\pi V_0} \sqrt{\rho_m K_s}
\end{equation}

Note that here we do need to know the adiabatic bulk modulus, at least roughly. Choosing value $\sim 50$ for a Lennard-Jones system, with $V_0\sim1000$ and $a \sim \rho_m \sim 1$ gives $\tilde\alpha \sim 2 \times 10^{-3}$. This does not quite match the value chosen by Gr{\o}nbech-Jensen and Farago, who used a value $10^{-4}$ or a factor of 20 smaller, implying an underdamped dynamics (with a quality factor of 20), although the effective damping was probably smaller due to coupling with the particle dynamics. A recommendation would thus be to use Eq.~(\ref{tilde_alpha}) as an upper bound for $\tilde\alpha$ and experiment with reducing it by a factor of order 10.

\subsection{IntegratorNVU}
This is a new molecular dynamics that instead of conserving the energy $E$ (as in Newtonian dynamics) conserves the total potential energy $U$. NVU dynamics 
traces out a geodesic curve on the so-called constant-potential-energy hypersurface $\Omega$, given by ($\textbf{R} \equiv (\textbf{r}_{1},...,\textbf{r}_{N})$)

\begin{equation}\label{omega_def}
  \Omega = \{ \textbf{R} \in R^{3N} \,\ | \,\ U(\textbf{R})=U_0 \},
\end{equation}
in which $U_{0}$ is a constant. More details on the dynamics and its implementation can be found 
in T. S. Ingebrigtsen \textit{et al.}, J. Chem. Phys. 135, 104101 (2011); ibid. 135, 104102 (2011); ibid. 137, 244101 (2012). Examples of use are 
given below

\begin{verbatim}
# NVU dynamics with (mass-weighted) step size in 3N-dimensional 
# configuration space dispLength = 0.01 and potentialEnergyPerParticle = -1.0
itg = rumd.IntegratorNVU(dispLength=0.01, potentialEnergyPerParticle=-1.0)
\end{verbatim}
NVU dynamics gives equivalent results to NVE dynamics 
for most properties, and \verb|potentialEnergyPerParticle| could, for instance, be chosen as the average potential energy of a given state point. There is no 
drift of the potential energy as
this is explicitly corrected by the implementation. \verb|dispLength| (also 
known as $l_{0}$) is similar to the time step $\Delta t$ of NVE dynamics and 
cannot be chosen arbitrarily large.

\subsection{IntegratorSLLOD}

This integrator uses the SLLOD equations of motion are used 
to simulate simple shear flow. The flow is in the $x$-direction with a gradient
in the $y$-direction; that is, the further you go in the (positive) 
$y$-direction, the faster the streaming velocity (in the positive 
$x$-direction). As is typically done we use an isokinetic thermostat, which 
keeps the kinetic energy constant. Our implementation is based on Pan et al., 
``Operator splitting algorithm for isokinetic SLLOD molecular dynamics'', J. 
Chem. Phys., 122, 094114 (2005), and conserves the kinetic energy to a high 
precision. Note that while RUMD in general is coded in single precision, parts of SLLOD
are done in double precision. These include the kinetic energy-like quantities
which are summed for the isokinetic thermostat. In addition the incrementing
of the box-shift is done in double precision on the host in order to allow
small strain rates.

This integrator requires a simulation box that implements a special kind of
periodic boundary conditions, so-called Lees-Edwards boundary conditions, 
implemented using \verb|LeesEdwardsSimulationBox|. Here is the 
Python code for running a SLLOD simulation:

\begin{verbatim}
sim = Simulation(...)

# create a Lees-Edwards box based on the existing box
# (can be omitted if using a configuration that came from a 
# SLLOD simulation)
le = rumd.LeesEdwardsSimulationBox(sim.sample.GetSimulationBox())
sim.SetSimulationBox(le)

# strain rates below 1.e-6 may not be stable for small box sizes (the box-shift
# per time step must not be swallowed by round-off error)
itg = rumd.IntegratorSLLOD(timeStep=0.004, strainRate=1.e-5)
sim.SetIntegrator(itg)

# T is the desired temperature; N the number of particles
# -3 due to momentum conservation
# (perhaps should be -4 because of the kinetic energy constraint)
itg.SetKineticEnergy(0.5*T*(3*N-3.))

# xy is the relevant component of the stress (from version 3.0 includes kinetic contribution by default)
# for stress-strain curves, determining viscosity, etc.
sim.SetOutputMetaData("energies", stress_xy=True)

sim.Run(...)
\end{verbatim}
Note that the analysis tools that analyze configurations (such as 
\verb|rumd_rdf| and \verb|rumd_msd|) correctly handle the Lees-Edwards boundary
conditions. It is important to realize, however, that the \verb|rumd_msd|
only considers displacements transverse to the shearing ($x$) direction.

\subsection{IntegratorMolecularSLLOD\label{IntegratorMolecularSLLOD}}

The above integrator, \verb|IntegratorSLLOD|, is for use on atomic systems. It
can be used with molecular systems, but there are some issues which arise with such 
systems, in particular to do with the thermostatting mechanism. To avoid these issues
there is a separate integrator class, \verb|IntegratorMolecularSLLOD|, which integrates
the molecular formulation of the SLLOD equations of motion. The essential
difference is that (1)
the streaming velocity used to adjust the changes in position and momentum 
is evaluated at the center of mass of the molecule containing that atom, and (2) the
thermostat constrains the kinetic energy associated with the molecular centers of mass,
leaving rotations and other degrees of freedom unthermostatted. The algorithm is a
straightforward generalization of the one of Pan et al that was used for the atomic
case. Use of \verb|IntegratorMolecularSLLOD| is exactly as for \verb|IntegratorSLLOD|;
note that it will give an error if no molecules have been defined 
(section~\ref{molecules}).


\subsection{IntegratorIHS\label{IntegratorIHS}}


Potential energy minimization is performed with \verb|IntegratorIHS| via 
Newtonian dynamics (Leap-frog algorithm). A path is followed in configuration
space where $ \sum_{i}\textbf{F}_{i} \cdot \textbf{v}_{i} > 0$. If the latter sum
becomes negative $\textbf{v}_{i}$ is set equal to zero for all particles $i$. A
simulation is run with

\begin{verbatim}
# timeStep specifies the time step of Newtonian dynamics
itg = rumd.IntegratorIHS(timeStep=0.0005)
\end{verbatim}
The \verb|timeStep| should be chosen smaller than usual MD simulations.
After setting the integrator on the simulation object one can call
\verb|Run(...)| in the usual way to integrate for a fixed number of steps.
A more typical mode of use is to run until some convergence criterion has been
reached. Moreover, it is also typical to be running an ordinary MD simulation,
and periodically make a copy on which energy minimization is performed. This
can be done by using the \verb|IHS_OutputManager| class, described in 
Section~\ref{other_output_managers}, which acts as a wrapper around 
\verb|IntegratorIHS|.


%itg.DumpInherentStateTransitionConfigurations()
%This command causes the last configuration to be stored in a specific directory. The default output directory for this integrator is 
%\verb|inherentTransitionFiles| (can currently not be changed). 

\subsection{IntegratorMMC}
\verb|IntegratorMMC| implements the Metropolis algorithm for canonical ensemble Monte Carlo simulations [N. Metropolis et al., J. Chem. Phys. 21, 1087 (1953)]. 
The integrator applies cubic-shaped all-particle trial moves and uses a random-number generator suitable for running efficient simulations 
on the GPU [GPU Gems 3, Chapter 37]. The integrator may be invoked with

\begin{verbatim}
# canonical ensemble Monte Carlo simulations
# dispLength = 0.01 is the sidelength of cubic trial displacements
# all particles are moved at once
itg = rumd.IntegratorMMC(dispLength=0.001, targetTemperature=1.0)
\end{verbatim}
The acceptance rate of the  MC algorithm can be extracted via

\begin{verbatim}
sim.Run(...)

# a magic number for the acceptance rate is 50%
ratio = itg.GetAcceptanceRate() 
print("MC acceptance ratio: " + str(ratio))
\end{verbatim}

\section{Molecules\label{molecules}}

The setting up of simulations involving molecules is described by a basic example in the tutorial. 
Here we add some further details about the intra-molecular potentials and show how to set up position and topology files for a mixture of molecules.

\subsection{Potentials for bonds, bond angles, dihedral angles}

The potential describing intra-molecular interactions in RUMD contains three terms: one for
bonds, one for bond angles and one for dihedral angles.
The bonds can be described either by a harmonic potential or by the Finite Extensible Nonlinear Elastic (FENE) potential.
The potential for harmonic bonds reads:
\begin{equation}
U_{\textind{bond}}(\vec{r}) =  \frac{1}{2}\sum_{\textind{bonds}}k_s^{(i)} (r_{ij}-l_b^{(i)})^2,
\end{equation}
where $k_s^{(i)}$ is the spring constant of bond type $i$ and $l_b^{(i)}$ its zero force bond length.
The potential for bond angles is given by:
\begin{equation}
U_{\textind{angle}}(\vec{r}) = \frac{1}{2}\sum_{\textind{angles}}k_\theta^{(i)} [\cos(\theta)-\cos(\theta^{(i)})]^2, 
\end{equation}
where $k_\theta^{(i)}$ is the angle force constant for angle force type $i$ and
$\theta_0^{(i)}$ the zero force angle.
The parameters for the harmonic bond and bond angle potentials come from the Generalized Amber Force Field
(GAFF) in Wang et al., ``Development and testing of a general amber
force Field'', J. Comp. Chem., 25, 1157 (2004).
The potential for the dihedral angles have the following form:
\begin{equation}
U_{\textind{dihedral}}(\vec{r}) = \sum_{\textind{dihed}} \sum_{n=0}^5 c_n^{(i)} \cos^n(\phi),
\end{equation}
where $c_n^{(i)}$ are the torsion
coefficients for torsional force type $i$. The parameters for the dihedral angle potential
come from Ryckaert et al., ``Molecular dynamics of liquid alkanes'', Faraday Disc. Chem. Soc., 66, 95(1978).

The FENE potential can also be used to describe bonds. It is given by:
\begin{equation}
U(\vec{r}) = -\frac{1}{2}kR_0^2\sum_{\textind{bonds}} \ln\left[ 1 -\left(\frac{r_{ij}}{R_0}\right)^2\right],
\end{equation}
where $k=30$ and $R_0=1.5$ (in reduced units). At the moment
the constraint method is applicable for molecules with few
constraints.


Here is a typical python code for setting up the intra-molecular potential:
\begin{verbatim}
[set up simulation object and add non-bonding potentials as usual]

# read topology file
sim.ReadMoleculeData("start.top")

# create intramolecular potential objects
pot_harm = rumd.BondHarmonic() # or rumd.BondFENE()
pot_angle = rumd.AngleCosSq()
pot_dihedral = rumd.DihedralRyckaert()

# set their parameters
pot_harm.SetParams(bond_type=0, bond_length=0.39, stiffness=2700, exclude=True)

pot_angle.SetParams(angle_type=0, theta0=2.09, ktheta=537)
pot_angle.SetParams(angle_type=1, theta0=1.85, ktheta=413)

pot_dihedral.SetParams(dihedral_type=0, coeffs=[0, 133, 0, 0, 0, 0])
pot_dihedral.SetParams(dihedral_type=1, coeffs=[0,-133, 0, 0, 0, 0])
pot_dihedral.SetParams(dihedral_type=2, coeffs=[15.5000,  20.3050, -21.9170, \
                                        -5.1150,  43.8340, -52.6070])
# add the potentials
sim.AddPotential(pot_harm)
sim.AddPotential(pot_angle)
sim.AddPotential(pot_dihedral)
\end{verbatim}
Note that there can be as many bond-, angle- and dihedral- types as wished. The function \verb|DihedralRyckaert.SetParams()| has two arguments: the dihedral angle type $i$ and a list or array of values of the torsion coefficients from $c_0^{(i)}$ to $c_5^{(i)}$. From version 3.1 it does not matter whether the non-bonding interactions are specified first or the intra-molecular interactions. Prior to that it was necessary to specify first the non-bonding interactions, otherwise the exclusions would not be correctly applied. The way to obtain the topology file \verb|start.top| used above will be described in section~\ref{sec:topology}
below.

The argument \verb|exclude| to \verb|SetParams| on the harmonic bond indicates whether the existing interactions (likely some variant of ranged interaction like Lennard-Jones) should be excluded.  This argument should generally be \verb|True| for harmonic bonds, and \verb|False| for FENE bonds.

Finally there are alternative angle and dihedral-angle potentials. The respective classes are named \verb|AngleSq| and \verb|PeriodicDihedral|. The form of the potential for  \verb|AngleSq| is a simple harmonic potential in the angle, ${\textstyle \frac{1}{2}} k (\theta-\theta_0)^2$, so the parameters are analogous to those of the \verb|AngleCosSq| case. For \verb|PeriodicDihedral| the interaction energy is a function with four rather than six parameters. NEED REFERENCE TO GET FUNCTIONAL FORM.

\subsection{Constraints on bond lengths}

As an alternative to using intra-molecular potentials, molecules can be 
simulated using constraints, whereby a fixed distance between two particles is 
maintained via a constraint force. The actual implementation is described in 
[S. Toxvaerd \textit{et al.}, J. Chem. Phys. 131, 064102 (2009)]. 
The procedure for setting up a rigid bond between two particles is analogous to 
adding a bond potential between the two particles, i.e.,

\begin{verbatim}
# the distance between all bonds of type = 0 is length = 0.584
cons_pot = rumd.ConstraintPotential()
cons_pot.SetParams(bond_type=0, bond_length=0.584)
sim.AddPotential(cons_pot)
\end{verbatim}
The constraint forces are calculated by solving a set of non-linear equations. This is achieved by first making them linear and thus involves iteration. The number of times the 
linear set of equations is iterated can be controlled with

\begin{verbatim}
# iterate the linear set of equations 5 times
cons_pot.SetNumberOfConstraintIterations(5)
\end{verbatim} 
The more iterations the more costly the algorithm becomes. Since there is no convergence check, one should verify that the bond lengths are satisfied to within the desired tolerance via

\begin{verbatim}
sim.Run(...)

# prints the standard deviation of all constrained bond lengths
# with respect to the length values
pot_cons.WritePotential(sim.sample.GetSimulationBox())
\end{verbatim} 
A good value for the standard deviation would be less than $10^{-5}-10^{-6}$. More detailed statistics on the bond length distribution may be obtained from the post-processing tool \verb|rumd_bonds| (described below)

Due to an internal constraint on the linear equation solver, there may not be more than 512 constraints (fixed bonds) per molecule. This will be lifted in a future RUMD release. It should also be noted that for each bond constraint one degree of freedom is naively removed (for the purpose of evaluating the kinetic temperature). The implementation does not try estimate whether the system is, for instance, over-specified, but the user can manually specify the number of degrees of freedom via \verb|sim.sample.SetNumberOfDOFs()|. 

For linear molecules, for instance chain molecules without side groups which are often used as course-grained models for polymers, an optimized linear solver has been implemented. This solver can be chosen by calling
\begin{verbatim}
# use optimized solver for linear molecules with constraints
cons_pot.SetLinearConstraintMolecules(True)
\end{verbatim}
after creating the \verb|ConstraintPotential| object. This optimization is possible because the system of linear equations for a linear molecule has a simple tridiagonal shape. For this optimized solver to work properly, the bonds in the topology file have to be ordered. Moreover, since the same solver is used for all molecules in the simulation, the solver only gives correct results if \emph{all} molecules containing constraints are linear. This optimization has not been tested for a large range of molecule sizes, but for Lennard-Jones chains consisting of 9 constraint bonds, the simulation time is about $70\%$ of the default linear solver time.

\subsection{The topology file}

An example of topology file can be found in the
subdirectory \verb|Conf/Mol|. It is called \verb|mol.top|. It is
a text file which specifies the bonds between the atoms.
To define the bonds in a \verb|.top| file we:
\begin{enumerate}
\item Specify the beginning of the bond section with the keyword
\begin{verbatim}
[ bonds ]
\end{verbatim}
We will call this a section specifier.
\item Add a comment line starting with \texttt{;}, for example,
\begin{verbatim}
; My bonds
\end{verbatim}
\item Specify the bonds.
  Each bond is described row-wise using four positive entries which
  are all integers:
  (i) The first entry is the the molecule index, (ii) the second
  and third integers are the indices of the particle forming
  the bond and (iii) the forth integer sets the bond type. This means
  that if atom 42 is bonded with 55 in molecule 2 with bond type 5 we
  write
\begin{verbatim}
2 42 55 5
\end{verbatim}
and so forth.
\end{enumerate}

Likewise for angles. Here the section specifier is \verb|[ angles ]|
and five entries are needed to specify the angle. The first again
indicates what molecule this angle belongs to and the last
specifies the angle type. For example, if
atom with index 42 is the vertex of the angle and atoms 55 and 78 are
the two end-points we can use
\begin{verbatim}
2 55 42 78 0
\end{verbatim}
This angle is found in molecule with index 2 and is of type 0. As you
might have guessed, to specify a dihedral angle you need 6 entries
since a dihedral is defined via three bonds. The section specifier is
\verb|[ dihedrals ]|.

Here is a simple example on how to write a topology file of a
single butane molecule based on a united-atomic-unit model,
where the methyl (the two $\mbox{CH}_3$ groups) and methylene
(the two $\mbox{CH}_2$ groups) are represented by a single particle:
\begin{verbatim}
[ bonds ]
; Bond section - ks = 311 kcal/mol lb = 3.4 ?~E
0 0 1 0
0 1 2 0
0 2 3 0

[ angles ]
; Angle section - ktheta = 63.7 kcal/(mol rad**2) theta0 = 109 degr.
0 0 1 2 0
0 1 2 3 0

[ dihedrals ]
; dihedral/torsion section - RB coefficients 
0 0 1 2 3 0
\end{verbatim}

\subsection{Tools for setting up position and topology files}
\label{sec:topology}

Writing position (\verb|.xyz|) and topology (\verb|.top|) files from scratch for a mixture of molecules can be a lot of
work and it is very error prone to type in all the
information. Fortunately, RUMD provides a useful tool to do that
from the position and topology files of single molecules:

\begin{description}
\item[rumd\_init\_conf\_mol]
  This program allows you to set up the configuration for a molecular
  system (including molecular mixtures and mixtures of molecules and atoms) from \texttt{.xyz} and
  \texttt{.top} files of the individual molecules. The molecules are put 
  by default on a cubic lattice with the appropriate spacing. The files
  \texttt{start.xyz.gz} and \texttt{start.top} are written and can be used as
  start files. Usage:
\begin{verbatim} 
rumd_init_conf_mol [--seed N] [--temperature X] [--density X] [--norandomorientation] \\
[--spacing X] <m1.xyz> <m1.top> <num1> ...
\end{verbatim}
  \begin{description}
  \item[] {The arguments in square brackets are optional.}
  \item[\texttt{--seed N}:] {to give the seed (integer). The default value is 4622.}
  \item[\texttt{--temperature X}:] {to give the temperature (float). The default value is 1.0.}
  \item[\texttt{--density X}:] {to give the density (float). The default value corresponds
  to a spacing of twice the size of the largest molecule.}
  \item[\texttt{--norandomorientation}:] {to ask for not randomizing the molecules orientation. By default,
  the orientations are random.}
  \item[\texttt{--spacing X}:] {to give the spacing between two lattice sites (float).
  The default value is twice the size of the largest molecule.
  If both density and spacing are given by the user,
  that corresponding to the largest spacing is used.}
  \item[\texttt{m1.xyz}:] {\texttt{.xyz} file for the atoms in molecule one (string).}
  \item[\texttt{m1.top}:] {the topology file for molecule one (string). If this is an atom, this string should begin with
  the word "atom". The atom topology file, which would be empty, does not need to exist. Files containing both molecules and isolated atoms can be created, although RUMD does not yet support simulations of such systems.}
  \item[\texttt{num1}:] {the number of molecules of this type (integer).}
  \item[] {and so forth.}
  \end{description}
  The atom velocities are chosen from a uniform distribution with zero mean and a variance
  corresponding to the desired temperature.
  The mass of the atom types involved in each molecule type
  can be given in the comment line of the \texttt{.xyz} file of each single molecule. For example,
  the \texttt{.xyz} file of
  a molecule containing atom type $0$ and $2$, with mass $1.0$ and $2.4$ respectively can have a
  comment line beginning by
  \begin{verbatim} 
  numTypes=2 mass=1.0,2.4
  \end{verbatim}
  If this line is not found in the \texttt{.xyz} file of the single molecule, the masses of the atom types
  declared in this molecule are set by default to $1.0$, unless the masses of the same atom types
  are declared in the \texttt{.xyz} files of other molecules.
  The molecule centers-of-mass are positioned on a low density simple cubic
  lattice, and you will most likely need to compress your system to a more appropriate density before running proper simulations. This can be done with the python module \verb|RunCompress| which gently compresses the system while letting the molecules interact with each other and move into more favourable positions.
\end{description}

%\noindent \textbf{\textit{Example}}

\begin{example}
Assume you wish to simulate a system composed a mixture of 200 FENE molecules
with 5 beads in each and of 100 atoms. You can write the single-molecule topology file 
of the FENE molecule as:

\begin{verbatim}
[ bonds ]
; Single FENE chain
0 0 1 0
0 1 2 0
0 2 3 0
0 3 4 0
\end{verbatim}
We here let both the molecule index and the bond type be 0, but they need not to be. This file can be saved as, for example, \verb|single_fene.top|. To specify the relative positions of the beads in the molecule
write a single-molecule \verb|.xyz| file
\begin{verbatim}
5
numTypes=1 mass=2.0 Single FENE
0 0 0 0
0 0 0 1
0 0 0 2
0 0 0 3
0 0 0 4
\end{verbatim}
This corresponds to a simple linear chain of five identical particles of type 0 equally spaced along the $z$-axis.
The mass of these particles is $2.0$.
This file can be saved as, for example, \verb|single_fene.xyz|.
There is no topology file for the single atom. The \verb|.xyz| file for the single atom of type $1$
and for which no mass is declared can look like:
\begin{verbatim}
1
single atom
1 0 0 0
\end{verbatim}
It can be saved as, for example, \verb|atom.xyz|.
To create the mixture of 200 molecules and 100 atoms:
\begin{verbatim}
./rumd_init_conf_mol single_fene.xyz single_fene.top 200 atom.xyz atom.top 100
\end{verbatim}
This command produces two files called \verb|start.top| and
\verb|start.xyz.gz| with all the information needed to start a
simulation. The mass of atom type $1$ is set to $1.0$. As mentioned above the density of the system is
probably not correct. To compress and equilibrate the system, a loop consisting
of a series of short simulations, increasing density by a small fraction each time, can be used. Such a loop is provided in the module \verb|RunCompress|, which is used as follows:

\begin{verbatim}
from rumd.RunCompress import RunCompress

# create sim object, read configuration, set potential(s) etc
... 

RunCompress(sim, final_density=0.9)
\end{verbatim}

Here \verb|final_density| is the desired density (in particles per units volume, not molecules per unit volume) that one wants to achieve before running a simulation proper. Other arguments to \verb|RunCompress| are \verb|num_scale_steps| (default 100) which is the number of times the system is uniformly scaled to get to the final density and \verb|niter_per_scale| (default 1000) which is the number of time steps that are run for each scaling step.

\end{example}

\section{Output}

The core of a molecular dynamics program is the algorithm, which needs to be as fast as 
possible. But the algorithm 
is useless without output, and the more control a user
has over the output the more useful the software. An additional consideration
is that with a very fast code one has to be careful not to drown in data, 
particularly where configurations are concerned. We use so-called {\em output
managers} to control both how frequently a given kind of data is written
({\em scheduling}) and precisely what data is included in the files (called the 
{\em meta-data}, a specification of what data is in a given output file).


\subsection{Output managers}

Output is controlled by {\em output managers}. These are software components 
(classes) written in C++ or Python which manage the writing of various 
observables to files. A given output manager manages
writing of data to one output file at a time, 
for example one containing energy-like quantities (energies
file), or one containing configurations (trajectory file). The basic output 
managers are written in C++ but can of course be controlled using the Python
interface. The mechanism for handling Python output managers exists in order
for users to be able to write their own run-time analysis code in Python 
without having to recompile the main RUMD software (i.e. the C++ library).

We start by discussing the C++ output managers, in particular the so-called
energies output manager and trajectory output manager.
In order to avoid data corruption in the event of a long simulation being 
interrupted, for the purposes of output the simulation run is divided into
output ``blocks''. The output manager starts a new file each time a new block
is started, with an incremented block index, thus energies files have names
like \verb|energies0000.dat.gz|, \verb|energies0001.dat.gz|, etc. and 
trajectory files have names \verb|trajectory0000.xyz.gz| etc. (previously, up to
version 2.0, \verb|block0000.xyz.gz|: old data is still be readable). Note that
the files
are written in ASCII to be human readable but then compressed using gzip. 
These files reside by default in the directory \verb|TrajectoryFiles|, but for 
standard analysis and 
post-processing it is not necessary to be aware of them (the analysis tools
know where to find them). On the other hand it is straightforward to make
a concatenated uncompressed energies file using 
\verb|zcat TrajectoryFiles/energies* > energies.dat| if needed. The
 uncompressed file will take up 2-3 times as much disk space as the 
compressed ones did. As much as possible, analysis scripts should also read 
directly the compressed files
(in Python this is easily done using the \verb|gzip| module) rather than 
requiring the user to separately uncompress the files.


The output files appear in the directory \verb|TrajectoryFiles| by default. If
this directory exists it is renamed with a \verb|.rumd_bak| extension (if the
latter exists it is overwritten). From version 2.1.1 on it is possible to
disable backup by calling  \verb|sim.sample.EnableBackup(False)|.
Changing the output directory is done as follows:

\begin{verbatim}
sim.SetOutputDirectory("TrajFilesRun01")
\end{verbatim}
Data files in general should be self-describing; in particular it should be 
possible to look at the file and know what each column represents. We implement
this principle in both energies and trajectory files. Analysis programs should
make use of the provided meta-data instead of expecting a certain order of
columns. 

If not explicitly set by the user via \verb|sim.SetBlocksize()|, the
block size will be 
chosen automatically when \verb|sim.Run()| is called, such that the number of
blocks is of order 100-1000 and the block size a power of 2.

\subsection{Log-lin scheduling: Linear, logarithmic, limlin and mixed.}

The output managers can save items (configurations to a trajectory file or
lines to an energies file) at equally spaced intervals (linear saving), 
logarithmically spaced (in powers of two, plus the zeroth) or a ``mixed'' 
scheduling called log-lin. The scheduling for a given output manager is
set via
\begin{verbatim}
SetOutputScheduling(manager, schedule, **kwargs)
\end{verbatim} 
which was described in the tutorial. The \verb|manager| argument is the name of
an output manager, for example ''trajectory'' or ``energies''. The 
\verb|schedule| argument is one of ''linear'' (which requires a following
keyword argument \verb|interval|), ''logarithmic'' (which can take an optional
following keyword argument \verb|base|), ''none'', ''loglin''
(which requires keyword arguments \verb|base| and \verb|maxInterval|), or ''limlin'' (which requires keyword arguments \verb|interval| and \verb|numitems|). Since 
''linear'' and ''logarithmic'' scheduling are special cases of ''loglin'', let
us consider the meaning of the parameters \verb|base| and \verb|maxInterval|:

\begin{description}
\item[base] This is the smallest interval of time steps considered. 
All intervals between writing items are multiples of base.
\item[maxInterval] The largest interval allowed between saves. Its purpose is 
to allow combined linear and
logarithmic saving. It should be set to zero to do pure logarithmic saving.
If non-zero, the interval between writes will, after increasing initially
logarithmically, become fixed at maxInterval.
\end{description}

Some possible combinations are, assuming blockSize=128:

\begin{description}
\item[Pure logarithmic saving with base 1] Set base=1, maxInterval=0, and you
get time steps 0, 1, 2, 4, 8, 16, 32, 64, 128 within each block. This is 
equivalent to ``logarithmic'' scheduling without specifying \verb|base|.
\item[Pure linear saving at interval 4 time steps] Set base=4 and maxInterval=4
and you get time steps 0, 4, 8, 12, 16, \ldots, 128. This is equivalent to
``linear'' scheduling with \verb|interval|=4.
\item[``log-lin''] Set base=1 and maxInterval=8, and you
get time steps 0, 1, 2, 4, 8, 16, 24, 32, 40, 48, 56, \ldots, 128. 
\end{description}


Note the following restrictions on the values of base
and maxInterval:  When maxInterval is non-zero it must be a 
multiple of base. Except for the case of linear saving (base=maxInterval), both
base and maxInterval must divide evenly into blockSize. 

The scheduling type ''limlin'' is not a special case og log-lin. It saves linearly spaced with specified interval, starting at the beginning of the block, but only until a total \verb|numitems| saves have been performed. Nothing more is saved until the start of the next block. This can be used when short-time linearly spaced is necessary, as well as averaging over longer times, without having to save the entire trajectory at short intervals.


The meta-data in the comment line of a trajectory file contains an item
\verb|timeStepIndex| which gives the number of the time step {\em within the
output block} and a string which looks like \verb|logLin=0,10,0,4,10|.
The five items in this list identify both the parameters of a particular 
log-lin saving scheme and the indexing of the current time step with that 
scheme. It can be useful occasionally to understand these parameters.
A description of each one follows, in the order they appear in the string:

\begin{description}
\item[block] Identifies which output block we are in. Will therefore be the same
for all items written in a given block.
\item[base] Smallest interval of time steps considered. It does not change 
during a simulation run, but is a parameter controlled (perhaps indirectly)
by the user.
\item[index] Labels items within a block, in that it starts at zero and 
increments by 1 for each item written. When maxInterval is set to zero
(pure logarithmic saving), there is a simple relation between index and 
nextTimeStep: nextTimeStep = 0 if index = 0 and nextTimeStep = 
base $\times$ $2^{\textrm{index-1}}$ otherwise. Note that the block size is given
(again for pure logarithmic saving, maxInterval=0) by base 
$\times 2^{\textrm{maxIndex-1}}$ 
\item[maxIndex] The value of ``index'' for the last item in the block. So if
the saving scheme is such that there are 8 items per block, maxIndex will have
the value 7 (since index starts at zero). Does not change during a simulation
run\footnote{The exception to this is linear saving where the interval does not evenly divide the blockSize.}; is set based on parameters specified by the user.
\item[maxInterval] The largest interval allowed between saves. This is a user
parameter which does not change. There is a more complicated 
relation between nextTimeStep and index in this case.
\end{description}


\subsection{Trajectories and energies}

Trajectory and energy files are the primary output of RUMD, so it is worth 
describing them in some detail. Trajectory files contain configurations, 
including the types and positions of all particles, and optionally their 
velocities, forces, and other quantities. Energy files consist of one line of
output for a given time step, which lists global quantities, such as the total
potential energy, or pressure, or a component of the stress tensor.
To keep track of potential changes in the 
formats for these files there is a format tag \verb|ioformat| which appears
in the meta-data (first line of each file). The current value of \verb|ioformat|
is 2. The difference between \verb|ioformat| 1 and 2 involves changes in the
meta-data for trajectory (configuration) files. The software and analysis can
handle either format transparently. 


Controlling what gets written, and the precision, in the energies and trajectory
files is done via \verb|SetOutputMetaData| as described in the tutorial. One
can specify more than one variable in a given call, for example:

\begin{verbatim}
sim.SetOutputMetaData("energies", totalEnergy=False, stress_xy=True)
\end{verbatim}
A look at the first few lines of an output file (using 
\verb^zcat TrajectoryFiles/energies* | head^, for example) shows the meta-data.
For the energies file and default settings, it might look like the following:

\begin{verbatim}
# ioformat=2 N=1000 Dt=0.005000 columns=ke,pe,p,T,Ps,Etot,W
0.78297 -6.92308 3.65423 0.522503 0 -6.14011 2.52269 
0.779256 -6.9193 3.66705 0.520024 0 -6.14004 2.53585 
0.77511 -6.91508 3.68272 0.517257 0 -6.13997 2.55168 
...
\end{verbatim}


The comment line contains the meta-data, including here the \verb|ioformat|, time interval between lines (for energies files one typically uses linear saving), 
number of particles (useful for analysis) and the 
list of symbols identifying the columns. 
It is not a good idea to rely too much on a particular
order of columns; when writing an analysis script in Python for example, it is
recommended to read the meta-data and extract the column information so that
the potential energy column, for example, is always correctly identified.



\begin{table}
  \caption{\label{energyQuantities}Identifier and column-label for the main
    quantities that can be written to the energies file. For extensive 
    quantities, such as potential energy or virial (though not volume), the 
    ``per-particle'' value is written.}
  \begin{center}
    \begin{threeparttable}
      \begin{tabular}{|c|c|}
        \hline
        identifier & column label \\
        \hline
        \verb|potentialEnergy| & \verb|pe| \\
        \verb|kineticEnergy| & \verb|ke| \\
        \verb|virial| & \verb|W| \\
        \verb|totalEnergy| & \verb|Etot| \\
        \verb|temperature| & \verb|T| \\
        \verb|pressure| & \verb|p| \\
        \verb|volume| & \verb|V| \\
        \verb|density| \tnote{(a)}& \verb|rho| \\ 
        \verb|thermostat_Ps| \tnote{(b)} & \verb|Ps| \\
        \verb|enthalpy| & \verb|H| \\
        \verb|stress_xx|\tnote{(c)} & \verb|sxx| \\
        \verb|stress_yy| & \verb|syy| \\
        \verb|stress_zz| & \verb|szz| \\
        \verb|stress_xy| & \verb|sxy| \\
        \verb|stress_yz| & \verb|syz| \\
        \verb|stress_xz| & \verb|sxz| \\
        \verb|v_com_x| & \verb|v_com_x| \\
        \verb|v_com_y| & \verb|v_com_y| \\
        \verb|v_com_z| & \verb|v_com_z| \\
        \verb|potentialVirial| \tnote{(d)} & \verb|pot_W| \\
        \verb|constraintVirial| \tnote{(d)}  & \verb|con_W| \\
        \verb|simulationDisplacementLength| \tnote{(e)} & \verb|dispLength| \\
        \verb|instantaneousTimeStepSq| \tnote{(e)}  & \verb|dt^2| \\
        \verb|euclideanLength| \tnote{(e)} & \verb|eclLength| \\
        \hline
      \end{tabular}
      \begin{tablenotes}
      \item[(a)] From version 2.1 on.
      \item[(b)] The extra degree of freedom used by the NVT integrator.
      \item[(c)] Versions before 3.0 did not include the kinetic contribution to the atomic stress by default. Change using (\verb|Sample| method) \verb|SetIncludeKineticStress(bool)|. Contributions to the atomic stress from bond-constraints, angle- and dihedral-forces are not included.
      \item[(d)]\verb|potentialVirial| : contribution to virial from the potential; \verb|constraintVirial| : contribution to virial from constraints.
      \item[(e)] Parameters associated with NVU dynamics (\verb|IntegratorNVU|)
      \end{tablenotes}
    \end{threeparttable}
    
    \end{center}
  \end{table}




For trajectories, we use an extended xyz format, and
a given file will contain all the trajectories from one 
output block. For logarithmic saving this will be a relatively small number of
configurations, less than 20. Each configuration starts, as is usual with the
xyz format, with the number of particles. The line following is considered a
comment line in the xyz format. We use it to write the meta-data. This can
include
information about the simulation such as the simulation box and the integrator
(including whatever parameters are needed to restart a simulation),
the time step and log-lin parameters and a list of symbols identifying the
columns. It can sometimes be necessary to look at the configuration file
(for example in order to check the number of particles or the box size).
Here is an example:

\begin{verbatim}
1000
ioformat=2 timeStepIndex=0 logLin=0,10,0,4,10 numTypes=2 integrator=IntegratorNVE,0.005 \
sim_box=RectangularSimulationBox,9.41,9.41,9.41 mass=1.0000,1.0000 \
columns=type,x,y,z,imx,imy,imz,pe,vir
0 4.6021 0.1670 3.7406 0 0 0 -7.4641 13.2556
0 -0.2562 -4.1929 4.5378 0 0 0 -6.9324 7.1333
0 -3.2822 2.1659 1.1286 0 0 0 -7.2647 15.1471
\end{verbatim}


\begin{table}
  \caption{\label{trajectoryQuantities}Identifier and column-label for the
    quantities that can be optionally written or not to the trajectory file.}
  \begin{center}
      \begin{tabular}{|c|c|}
        \hline
        identifier & column label \\
        \hline
        \verb|images| & \verb|imx, imy, imz| \\
        \verb|velocities| & \verb|vx, vy, vz| \\
        \verb|forces| & \verb|fx, fy, fx| \\
        \verb|pot_energies| & \verb|pe| \\
        \verb|virials| & \verb|vir| \\
        \hline
      \end{tabular}
      \end{center}
    
  \end{table}

The columns labelled \verb|imx| etc., give the image indices---what multiple of
the box size must be added in a given direction to give the total displacement
taking into account the number of times a periodic boundary has been crossed. 
It is essential for correct determination of mean squared displacement, for
example. Table~\ref{trajectoryQuantities} shows the quantities that can be
controlled using \verb|SetOutputMetaData| on the  \verb|trajectory| output
manager. So including  \Q{forces=True} for example in a call to  \Q{SetOutputMetaData}  causes the the individual particles' forces to be written to the trajectory configuration files.



\subsection{External calculators: adding data to the energies file}

In addition to including or excluding the standard energy-like quantities, it
is possible to attach an ``external calculator'' which knows how to calculate
some quantity, and have that quantity be included in the energies file. This
can be more convenient than creating separate files for new quantities.

\begin{verbatim}
pot18 = Pot_IPL_n_18(...)
alt_pot_calc = rumd.AlternatePotentialCalculator(sample=sim.sample, alt_pot=pot18)
sim.AddExternalCalculator(alt_pot_calc)
\end{verbatim}

The potential energy calculated by the IPL $n=18$ potential will then appear
in the energies files in the column designated by the ``ID string'' of the
potential. The latter can be obtained via \verb|pot18.GetID_String()|
and changed via \verb|pot18.SetID_String()|. Another external calculator 
calculates the hypervirial by numerical differentiation. Doing the following


\begin{verbatim}
hypervirCalc = rumd.HypervirialCalculator(sample=sim.sample, delta_ln_rho=0.0005)
sim.AddExternalCalculator(hypervirCalc)
\end{verbatim}
will cause columns labelled \verb|approx_vir| and \verb|approx_h_vir| to appear
in the energies file. 

These represent the finite difference approximations to the
virial and hypervirial, respectively. The virial is included in order to be 
able
to compare it to the true virial as a check. The second argument to the 
constructor of \verb|HypervirialCalculator| is the change in $\ln(\rho)$
(approximately the same as the fractional change in density) to be used for the
finite difference calculation.
Note: the above lines must appear after setting the potential! 


\subsection{Other output managers (C++)\label{other_output_managers}}

An example of an additional output manager implement in C++ is one
for computing inherent states (local minima of the potential energy function in
3N dimensions), using \verb|IHS_OutputManager|. 
Here each time the \verb|Update| function is called, the sample configuration is
copied and an energy minimization calculation is run on the copy using the 
integrator \verb|IntegratorIHS| (see Section~\ref{IntegratorIHS}). 
The energies are written to files \verb|ihsEnergiesXXXX.dat.gz|, while the 
minimized configurations are written to files \verb|ihsXXXX.xyz.gz|.
\begin{verbatim}
ihs_man = rumd.IHS_OutputManager(sample=sim.sample, timeStep=0.005)
ihs_man.SetMaximumNumberIterations(5000)
sim.AddOutputManager("ihs", ihs_man)
sim.SetOutputScheduling("ihs", "linear", interval=100)
sim.SetOutputMetaData("ihs", writeIHS_Config=False) # turn off writing of minimized configurations
sim.sample.AddOutputManager(ihs_man)
\end{verbatim}
As with other output managers the \verb|SetOutputScheduling| method of 
\verb|Simulation| is used to control frequency of output. Similarly \verb|SetOutputMetaData| sets whether the minimized configurations should be written or not.
Note also that the energy minimization algorithm has not been used very much and probably could be improved, for example by using conjugate-gradient minimization to speed convergence near the minimum. The use of single precision in RUMD is likely to limit the convergence, however.


\subsection{Python output managers}
% Arno
% Probably want to enforce common block size and include in the restart mechanism
% Is there a list somewhere of all functions in python that work on sample? I added some useful ones here.

Sometimes it is desirable to do data analysis during run-time. This is  for 
instance the case when doing post-production analysis would involve saving huge
 amounts of data. A python output manager is useful in this case, since it 
allows so-called call-back functions to be added to the output manager, 
converting simulation data to output.

To use a python output manager, a call-back function has to be defined. The 
call-back function should work on the sample object, and return a string with 
the data, for instance:
\begin{verbatim}
def momentum(sample):
    vel = sample.GetVelocities()
    P   = sum(vel)
    return "%4g %4g %4g " % (P[0], P[1], P[2])
\end{verbatim}
Then a new python output manager can be made, and the call-back function should 
be added to it:
\begin{verbatim}
sim.NewOutputManager("analysis", outputDirectory="TrajectoryFiles")
sim.SetOutputScheduling("analysis", "linear", interval = 100, blockSize = 10000)
sim.RegisterCallback("analysis", momentum, header = "Px, Py, Pz, ")
\end{verbatim}
This writes data files named \verb|analysis0000.dat.gz|, \verb|analysis0001.dat.gz| etc. to the same directory as the other output managers.

Note that \verb|RegisterCallback()| can be called multiple times on the same 
python output manager but with different call-back functions. The output and 
header strings of the different call-back functions are concatenated to a single
 line (do not use any end of line characters in these strings). 

The scheduling of the python output manager is set in the same way as for the 
C++ output managers. They do not enforce the block size to be the same, however
 (related to this, is that the
restart mechanism does not take account of Python output managers). To use a 
different block size for a python output manager, the optional keyword argument
\verb|blockSize| can be used.

There is another (useful) difference between the python and C++ output managers: when the schedule of a python output manager is set to ``none'', it is necessary to specify an interval. The schedule ``none'' is for for a python output manager the same as the schedule ``linear'', with the difference that no output is written to files. So the call-back functions that have been added to the python output manager are still called on the specified time steps. This is for instance useful for scaling the box or changing potential parameters step by step, or for doing time averaging of simulation data.

The call-back function can get simulation data from the sample object. Useful
 functions on the sample object include:
\begin{description}
\item[\Q{GetPositions(self)}] Returns the atom positions as a numpy array.
\item[\Q{GetImages(self)}] Returns the atom images as a numpy array.
\item[\Q{GetVelocities(self)}] Returns the atom velocities as a numpy array.
\item[\Q{GetForces(self)}] Returns the atom forces forces as a numpy array.
\item[\Q{GetPotentialEnergy(self)}] Returns the total potential energy.
\item[\Q{GetVirial(self)}] Returns the total virial.
\end{description}
A more extensive list of \verb|Sample| methods is given in section~\ref{Sample_methods}.



\section{Optimizing performance with the autotuner}

The basic usage of the autotuner is described in the tutorial. It is convenient
to distinguish between ``user parameters'' and ``internal parameters'' (also
called technical parameters). User parameters specify the kind of simulation
the user wants to run: the potential and its parameters, the integrator and its
parameters (time step and temperature for example), the number of particles
(also of each type)
and the density. The type of graphics card is also considered a user parameter,
though it is not necessarily something the user can choose explicitly. The 
autotuner's task is to find the optimal technical parameters for a given set
of user parameters. After the call to \verb|Tune| the simulation object is left
in the same state as before, but the internal parameters have been set to the
optimal values, which are also written to the file \verb|autotune.dat|. If the 
file exists already when the autotuner is run, it is checked to see if the user
parameters match the current ones. If so, the optimal technical parameters are
read from the file instead of going through the autotune-process again. (Note
that a different GPU type is sufficient to cause the autotuner to run again.)


\subsection{Technical parameters determining performance}

Here is a description of the technical parameters optimized by the autotuner,
mentioning also their default values and 
how they can be set ``manually'' by the user.

\begin{description}
\item[pb] The number of particles per thread-block, for neighbor-list 
generation, force calculation and integration step, default value 32. A value
for \verb|pb| can be specified by the user as a constructor argument to
\verb|Simulation|: \verb|sim=Simulation("start.xyz.gz",pb=32)|, or set
later (together with \verb|tp|) via \verb|sim.sample.SetPB_TP(pb, tp)|. 
\item[tp] The number of threads per particle, default value 4, for 
neighbor-list generation and force calculation (not integration). 
The total number of threads in a thread block is therefore 
\verb|pb| times \verb|tp|. It may also be specified as a constructor argument
to \verb|Simulation| or via \verb|sim.sample.SetPB_TP(pb, tp)|
\item[NB skin] The thickness of the extra shell (beyond the maximum cut-off
from the pair potential) searched when constructing the neighbor list. Default
value 0.3. If there is more than one pair potential then in 
principle they could have different skin values, but it is not clear that this
would be advantageous; the autotuner enforces a common value. To manually 
change the skin use \verb|pot.SetNbListSkin(skin)|
\item[Neighbor-list method] Which method is used to construct the neighbor-list. The main two possibilities are ``n2'' meaning an order-$(N^2)$ algorithm is used, which is generally optimal for small system sizes, and ``sort'' which involves sorting the particles in memory according to their spatial coordinates, allowing for an order- $(N)$ algorithm. Another possibility is ``none'' which means no neighbor-list is used. This is relevant only for very small systems, where the cutoff is almost as large as it can be without exceeding half the box size. To manually change the NB method use \verb|pot.SetNB_Method("n2")| or \verb|pot.SetNB_Method("sort")|.
\item[sorting scheme] Versions of RUMD prior to 3.0 used sorting differently, uncoupled to the building of the neighbor-list. This old-style sorting is still available, because it can help to spatially sort the particles at regular intervals when using the ``n2'' NB-method (typically for sizes just below when the ``sort'' NB-method becomes optimal). Different sorting schemes are available: whether to sort in 1 (X), 2 (XY) or 3 (XYZ) dimensions. To set the sorting scheme manually do \verb|sim.sample.SetSortingScheme(rumd.SORT_X)| for X sorting, etc (the options are \verb|rumd.SORT_X|, \verb|rumd.SORT_XY| and \verb|rumd.SORT_XYZ|).
\item[sort-interval] The number of time steps between sorting when using the old sorting procedure, default 200. Use \verb|sim.SetSortInterval(...)| to change. Setting the value to be 0 causes old-style sorting to be disabled (this is default since version 3.0).
\end{description}


\subsection{Fine control of the autotuner}

It can be useful to keep one or more parameters fixed while tuning with respect
to all the others. This is done by passing keyword arguments to the constructor
of the autotuner, such as

\begin{verbatim}
at = Autotune(pb=32, NB_method="sort", ...)
\end{verbatim}
For this purpose the names of the technical parameters need to match the 
variable-names used in the Python script itself. They are: \verb|pb|, 
\verb|tp|, \verb|skin|, \verb|NB_method|, \verb|sorting_scheme|, \verb|sort_interval|. 

Another optional keyword argument is \verb|ignore_file=True|, which will cause 
the autotuner to ignore an existing file \verb|autotune.dat|. In this case
it will still (over-)write that file after the tuning process is complete.


\section{Post-production analysis tools}



The supplied tools are both stand-alone C++ programs that can be called from
the command line, and (in many cases) 
Python modules that can be imported into your script
in order to do analysis on the fly (for example if a script involves many 
simulations at different state points and the data is not otherwise needed
afterwards one can save disk space by doing the analysis after each state
point.)
Use of the most important tools was described in the tutorial, although not
all details were given. Here we give a 
more complete description of their options and output files. There are is not
complete harmony of different tools; the ``common'' options listed below are not
available on all tools, for example. There should be some kind of convergence
in later versions of rumd. The output files generally start with a comment line
which identifies the columns, unless otherwise specified.

%% TO DO for analysis tools:
%% Maybe Usage message should include a list of output files
%% Fix rumd_vhc (gave nan's in one test I did!)

%% DECIDE ON AUTOCORRELATION VERSUS CORRELATION SCRIPTS

\subsection{Common command-line options}

\begin{description}
\item[Help]  -h gives a list of options for the tool. For example
\begin{verbatim}
[bead30] ~%rumd_rdf -h
Usage: rumd_rdf [-n <number of bins>] [-m <minimum time between configurations>]
[-f <first_block>] [-l <last_block>] [-v <verbose>] [-d directory] 
[-e <writeEachConfig>] [-p <particles per molecule>]
\end{verbatim}
\item[Restricting to part of the simulation run] Including \verb|-f 10| means
data is read from block 10 onwards. Including \verb|-l 20| means data from 
blocks after block 20 will not be read.
\item[Non-standard output directory] If data is not in the default directory 
\verb|TrajectoryFiles| (either because 
it's been renamed or because SetOutputDirectory was used to change it from the
start), use the \verb|-d <dir>| 
command-line option to specify where the analysis 
program should look for the data
\item[Verbosity] Including \verb|-v 1| allows various messages to be written
to standard output. \verb|-v 0| (the default) disallows (non-essential) output.
\item[Molecular quantities] In some tools calculating structural and dynamical
quantities, the number of particles per molecule
may (or must) be specified (by \verb|-p|\footnote{For some tools \Q{-m} was previously used for this.}). If this is greater than 1 (the default) then corresponding structural 
and dynamical quantities for the molecular center of mass are also computed.
\end{description}

\section{List of analysis tools}

\subsection{Analysis of the trajectories}
\begin{description}
\item[\Q{rumd_rdf}] Compute time averaged radial distribution function. Usage: 
\begin{verbatim}
rumd_rdf [-n <number of bins>] [-m <minimum time between configurations>]
 [-f <first_block>] [-l <last_block>] [-v <verbose>] [-d directory] 
[-e <writeEachConfig>] [-p <particles per molecule>]
\end{verbatim} 
Output file: \verb|rdf.dat|, whose first column is the pair-distance $r$, with
subsequent columns giving $g(r)$ for all pair combinations of atom types; 
for example if there are two types, then the columns are $r$, 
$g_{00}$, $g_{01}$, $g_{10}$, $g_{11}$ (note $g_{01}=g_{10}$). If the option 
\verb|-e 1| is specified, then a separate rdf file for each
configuration is written. The comment line in \verb|rdf.dat| gives the 
concentrations of the different atom types. The range of $r$ values is the box 
size in the $x$-direction (but note that for a cubic periodic box of side $L$ 
non-zero values  do not appear for $r$ greater than $L\sqrt{3}/2$).
A minimum (simulation) time (default 0) may be specified using \verb|-m|
when configurations have been saved logarithmically in order to reduce
 processing time and avoid too
closely spaced configurations receiving too much weight in the average.
\item[\Q{rumd_msd}] Basic dynamical correlation functions: intermediate 
scattering function, mean-squared displacement, non-Gaussian parameter.
Usage: 
\begin{verbatim}
rumd_msd [-h] [-p <particles per molecule>] [-f <first_block>] 
[-l <last_block>] [-d directory] [-e <extra_times>] [-s <subtract_cm>] [-v <verbose>] [-t <allow_type_changes>]
\end{verbatim}
Note that a file \verb|qvalues.dat| must be present, containing the 
$q$-value at which the self-intermediate scattering function is to be
calculated (one value for each atom type). New in version 2.1 is that 
the restriction to configurations saved logarithmically has been relaxed.
When the argument \verb|-e 1| is present, the functions are evaluated more
relative times (within a block), giving greater time resolution (but
possibly noisier curves). When the argument \verb|-s 1| is present, the center of mass motion is subtracted off, which is necessary when using Langevin dynamics.
Output files: \verb|msd.dat| (mean-squared displacement versus time for 
different atom types), \verb|Fs.dat| (self-intermediate scattering function 
versus time for different types ), \verb|FsSq.dat|, \verb|alpha2| (non-Gaussian
 parameter versus time for different types). Note that when Lees-Edwards 
boundary conditions are present, only transverse displacements are 
included in the calculation.
%\item[\Q{rumd_vhc}] Compute the Van Hove Correlation function. Usage: \begin{verbatim}rumd_vhc [-n <number of bins>] [-m <time between configurations>]
% [-f <first_block>] [-l <last_block>] [-d directory] [-v <verbose>] [-h]\end{verbatim}
%Output file: \verb|vhc.dat|, first column gives displacement values, subsequent
%columns give the probability to have moved a given displacement for different
%particle types.
\item[\Q{rumd_sq}] Calculate the static structure factor by Fourier-transforming
  the radial distribution function in \verb|rdf.dat|. No attempt to  handle the finite 
box-size, and hence integration range, intelligently is made; artifacts of this
(extra oscillations) are present in the output. Usage: 
\verb|rumd_sq <q_start> <q_final> <density>|. 500 values of $q$ ranging between 
the two user-supplied end-points are calculated. New in version 2.1 is handling
an arbitrary number of types.
Output files: \verb|Sq.dat| (first column: $q$, subsequent columns: $S_{AB}(q)$ 
for different pairs AB of types), \verb|qmax.dat| (location of maximum 
of $S_{AA}(q)$ for each type A, can be used as input to \verb|rumd_msd| by renaming to 
\verb|qvalues.dat|).

\item[\Q{rumd_bonds}] (New in version 2.1.1) Compute bond length probability 
densities. Usage: 
\begin{verbatim}
rumd_bonds [-h] [-n <number of bins>] [-m <minimum time between configurations>]
           [-f <first_block>] [-l <last_block>] [-d directory] [-t topologyfile]
           [-e <writeEachConfig>] [-v <verbose>] 
\end{verbatim} 
Output file: \verb|bonds.dat|, where the first column gives the bond lengths (the bins) and the rest of the columns contain the probabilities of the bond lengths for each type of bond specified in the topology file. The distributions have been normalized so that the surface under the distribution is 1. See \verb|rumd_rdf| for explanations of the other command line options.

\item[\Q{rumd_rouse}] Calculate Rouse modes autocorrelation functions and the orientational autocorrelation of the end-to-end vector of linear molecules. (see M. Doi and S. F. Edwards, The Theory of Polymer Dynamics, Oxford Science Publications, 1986)
Usage:
\begin{verbatim}
rumd_rouse [-h] [-p <particles per molecule>] [-f <first_block>] 
[-l <last_block>] [-d directory] [-v <verbose>]
\end{verbatim}
Output files:
\verb|rouse_X0Xt.dat| (autocorrelation functions of Rouse modes)
\verb|rouse_X0X0.dat| (variances of Rouse modes)
\verb|rouse_R0Rt.dat| (orientational autocorrelation of end-to-end vector)
\verb|rouse_R0R0.dat| (mean squared end-to-end vector and squared radius of gyration).
Note that this analysis tool only gives correct/meaningful output for systems with one kind of (linear) molecule. The data in the trajectory files should be ordered such that the particle data are in the same order as the particles in the chain. (The program does not read the topology file.)
\item[\Q{rumd_visualize}] Load one configuration from each block into the
molecular visualization program \verb|vmd|.
\item[\Q{rumd_image}] View a configuration using the ray-tracing software POV-ray.
\begin{verbatim}
rumd_image [-h] [-f <block index>] [-i <input configuration>]
   [-p <povray header>] [-o <output base filename>]  [-W <width of output image>]
   [-a <angle>] [-y <height>] [-r <radial distance>]
   [-L <position that camera should look at>]
\end{verbatim} 
Use \verb|rumd_image -h| for more usage information.
\end{description}

\subsection{Analysis of the energies}
\begin{description}
\item[\Q{rumd_plot_energies}] Read data into \verb|xmgrace| from the energies 
files as block data, and display one or more columns. The purpose of the tool 
is to uncompress the energies files into a temporary ASCII file for 
\verb|xmgrace| to read. Command-line arguments are passed to \verb|xmgrace|.
At least one argument of the form \verb|-bxy 0:2| is required to choose columns
to be plotted (against each other, or as here, against line number).

\item[\Q{rumd_stats}] Basic statistics (means, variances, covariances, drift)
  of the data in the energies files. Usage: 
\begin{verbatim}
rumd_stats [-h] [-f<first_block>] [-l<last_block>] [-v <verbose> ]
[-d<directory>] [-b base_filename]
\end{verbatim}
Written to standard output: total line count, then mean, variance and 
standard deviation for each column of data in the energies files. Any columns 
in the energies files which are not among the ``standard'' quantities, for
 example data computed by an external calculator, are flagged with a line 
``Non-standard column identifier: ...''. Output files: \verb|energies_mean.dat|
 (mean values), \verb|energies_var.dat| (variances), 
\verb|energies_mean_sq.dat| (mean squared values, useful for combining data 
from independent runs to get 
overall mean and variance), \verb|energies_covar.dat| (covariances between all
 pairs of columns in the energies files), \verb|energies_drift.dat| (linear 
regression estimate of the linear drift of each quantity with respect to line 
number; divide by the time interval between lines to get a proper rate).
The option \verb|-b base_filename|, available from version 2.1.1, allows you to
do basic statistics on a different set of files in the output directory. They 
should be numbered like energies files, just with energies replaced by a 
different basename, and each one should start with a comment line containing 
"columns=a,b,c" to identify the columns.

\item[\Q{rumd_autocorrelations}] Autocorrelation functions of the data in the energies files. Usage:
\begin{verbatim}
rumd_autocorrelations [-h] [-n] [-l <log_binning>] [-w <write_fraction>]
                      [-c <column_name> [<column_name> ...]]
\end{verbatim}
The autocorrelation functions of all columns are calculated and saved to \verb|autocorrelations.dat.gz| by default. The first column of the output file contains the time. With the option \verb|-c column_name| it is possible to choose for which columns in the energies files to calculate the correlation function. The name of the column as in the energies files should then be given. Only the first 0.01 fraction of the autocorrelation is written to the output file by default to save space. It is possible to change this with the \verb|-w write_fraction| option. Another way to reduce the output file size is by performing logarithmic data binning. This is done with the option \verb|-l log_binning|, where the number of bins per decade should be given. The option \verb|-n| is used to normalize the autocorrelation with the variance, so it is unity at time zero.

\item[\Q{rumd_crosscorrelation}] A crosscorrelation function of two columns in the energies files. Usage:
\begin{verbatim}
rumd_crosscorrelation [-h] [-n] [-l <log_binning>] [-w <write_fraction>]
                      <column_name> <column_name>
\end{verbatim}
The two names of the columns for which the crosscorrelation is calculated should be given. The names of the columns will be used for the output file. The command \verb|rumd_crosscorrelation pe W| for instance will result in an output file with the name \verb|crosscorrelation-pe-W.dat.gz|. The other command-line options are the same as for the autocorrelations.

\item[\Q{rumd_response}] Calculate the response function of one or two columns in the energies files. Usage:
\begin{verbatim}
rumd_response [-h] [-l <log_binning>] <column_name> [<column_name> ...]
\end{verbatim}
One or two names of the columns should be specified. The response function is then calculated as the Fourier transform of the derivative of the cross- or autocorrelation function. Again, the name(s) of the column(s) will be used in the output file, so if one column name is given, for instance \verb|rumd_response pe| than the output file is \verb|response-pe.dat|. By default, the data are smoothed with logarithmic data binning with 20 bins per decade. To turn off data binning, write \verb|-l 0|. The output file contains three columns with the angular frequencies, the real parts, and the imaginary parts of the response function.
\end{description}

\subsection{Calling analysis tools via the Python interface}
Some of the analysis tools can be accessed within Python through the
\verb|rumd.Tools| module. These are \verb|rumd_stats|, \verb|rumd_rdf| and
\verb|rumd_msd|. Before the analysis tools can be called, it is necessary to 
make sure the relevant output files are closed. This is done via 
\verb|sim.sample.TerminateOutputManagers()|. Here is some example code for
\verb|rumd_stats|

%% Have not mentioned the Molecular-CM related stuff here

\begin{verbatim}
    import rumd.Tools
    ...
    sim.Run(n_run_steps)
    sim.sample.TerminateOutputManagers()
    # create a rumd_stats object
    rs = rumd.Tools.rumd_stats()
    rs.ComputeStats()
    meanVals = rs.GetMeanVals()
    meanSqVals = rs.GetMeanSqVals()
    pressure = meanVals["p"] # the keys are the symbols in the file meta-data
    pe = meanVals["pe"]
    pe_var = meanSqVals["pe"] - pe**2
    # to write the file that the command-line rumd_stats program writes:
    rs.WriteStats()
    # to print the stats summary to standard output:
    rs.PrintStats()
\end{verbatim}
and here is some code for \verb|rumd_rdf|:

\begin{verbatim}
    sim.sample.TerminateOutputManagers()

    rdf_obj = rumd.Tools.rumd_rdf()
    # constructor arguments: number of bins and minimum time
    rdf_obj.ComputeAll(1000, 100.0)
    # include the state point information in the rdf file name
    rdf_obj.WriteRDF("rdf%5.3f_%5.3f.dat" % (rho, T))
    # to get as numpy array, first the r-values
    rVals = rdf_obj.GetRadiusValues()
    # and then the actual g(r) values for type 0-type 1 pairs
    gVals = rdf_obj.GetRDFArray(type1=0,type2=1)
\end{verbatim}
and for \verb|rumd_msd|:

\begin{verbatim}
    sim.sample.TerminateOutputManagers()
    msd_obj = rumd.Tools.rumd_msd()


    msd_obj.SetQValues([7.25, 5.75]) # set qvalues (new in V2.1.1, otherwise
    # reads from the file qvalues.dat)
    msd_obj.SetExtraTimesWithinBlock # include extra time differences within
    # an output-block (new in V 2.1.1; corresponds to -e argument)

    msd_obj.ComputeAll()

    # get a nDataPoints x 2 numpy array containing times and mean squared
    # displacement values for  particles of type 0
    msd0 = msd_obj.GetMSD(0)
    # get the intermediate scattering function for particles of type 0
    isf0 = msd_obj.GetISF(0)
    # get the non-Gaussian parameter alpha2 for particles of type 0
    alpha2_0 = msd_obj.GetAlpha2(0)
    msd_obj.GetChi4(0) # get the variance of the intermediate scattering 
    # for particle type 0 (new in version 2.1)

    # write data to a file (all types)
    msd_obj.WriteMSD("msd.dat")
    msd_obj.WriteISF("Fs.dat")
    msd_obj.WriteAlpha2("alpha.dat")
\end{verbatim}

The module \verb|Python/analyze_energies.py| (new in version 2.1.2) defines a class \verb|AnalyzeEnergies| for the analysis of the energies. Methods for the calculation of correlation functions and response functions are implemented. The module also defines some generic functions for calculating correlation and response functions of any 1D array. Some example code:

\begin{verbatim}
    import rumd.analyze_energies as analyze
    # Create analyze_energies object, read relevant columns from energy files
    nrgs = analyze.AnalyzeEnergies()
    nrgs.read_energies(['pe', 'W'])
    
    # calculate the normalized autocorrelation
    time, auto = nrgs.correlation_function('pe', normalize=True)
    # calculate the first 10% of the crosscorrelation
    length = int(1.1 * len(nrgs.energies['pe']))
    time, cross = nrgs.correlation_function('pe', 'W', length=length)
    # calculate response function
    omega, response = nrgs.response_function('W', 'W')
    
    # Save correlation functions to file
    import collections
    output = collections.OrderedDict()
    output['time'] = time
    output['pe'] = auto
    output['pe-W'] = cross
    analyze.write_columns_to_file("mycorrelations.dat.gz", output)
    # Do logarithmic data binning on the response function and save to file
    output = collections.OrderedDict()
    bins = analyze.create_logarithmic_bins(omega[1], omega[-1], bins_per_decade=12)
    binned_omega, binned_response = analyze.data_binning(omega, response, bins)
    output['omega'] = binned_omega
    output['real'] = binned_response.real
    output['imag'] = binned_response.imag
    analyze.write_columns_to_file("myresponse.dat", output)
\end{verbatim}

\section{Additional features/tricks}


\subsection{Changing particle masses}

As a new feature in version 3.1 it is possible to change the mass of a particle type directly from Python (rather than hacking the configuration file). This is achieved via the \verb|Sample| methods\verb|SetMass| and \verb|SetMasses|:

\begin{verbatim}
sim.sample.SetMass(0, 4.5) # set mass for particle type 0 to be 4.5
sim.sample.SetMasses(numpy.array([3.6,1.6,0.8])) # set masses for all types using a numpy array (size must match the number of types)
\end{verbatim}

\subsection{Momentum resetting}

Due to the inevitable numerical round-off error due to finite precision
arithmetic (single precision in the case of rumd), it is necessary to 
periodically reset the center of mass velocity to zero. By default this is
done every 100 time steps. This interval can be changed via
\begin{verbatim}
sim.SetMomentumResetInterval(1000)
\end{verbatim}
and turn off entirely by passing zero to the above method.

\subsection{\label{runtime_actions}User defined run-time actions}

New in version 2.1 is a facility for allowing users to include extra steps in
the main integration loop. It operates similar to python output managers in
that the user supplies a function or bound 
method and specifies how frequently it should be
called. The method should not take any arguments so typically it will be a 
method bound to an object of a Python-class defined in the user's script,
which contains references to the sample or simulation or some other 
relevant object (such as a potential). 
For example if the sample should be gradually compressed from a low 
density (this might be done to prepare a molecular configuration), then we 
could
define a class which has a reference to the sample object, and a method
\verb|Compress|, as follows

\begin{verbatim}
class Compressor:
    def __init__(self, sample, scaleFactor):
        self.sample = sample
        self.scaleFactor = scaleFactor
    def Compress(self):
        self.sample.IsotropicScaleSystem(self.scaleFactor)

num_scale_intervals = 100
density_factor = rho_desired/rho_start
scale_factor = math.exp(-math.log(density_factor)/3/num_scale_intervals)
comp = Compressor(sim.sample, scale_factor)

scale_interval=1000
sim.SetRuntimeAction("compress", comp.Compress, scale_interval)
sim.Run(scale_interval * num_scale_intervals)

\end{verbatim}

The run-time actions are called in order they were added, before the force
calculation in the main loop. The list always includes sorting the particles 
in memory and setting the total momentum to zero. Weak references are used to
store the list of run-time actions in order to allow them to refer to the 
sample or simulation object without causing problems with circular references.

\subsection{Changing the density}

Here is a function that will change the simulation box in order to give a 
specified density.

\begin{verbatim}
def SetDensity(sim, new_rho):

    # scale to get right density
    nParticles = sim.GetNumberOfParticles()
    vol = sim.GetVolume()
    currentDensity = nParticles/vol
    scaleFactor = pow(new_rho/currentDensity, -1./3)
    sim.ScaleSystem(scaleFactor)
\end{verbatim}

When simulating molecules, it might be more reasonable to keep the distances within the molecules fixed, for instance when simulating rigid bonds. In this case, use the command \verb|sim.ScaleSystem(scaleFactor, CM=True)|.

\subsection{Instantaneously scaling the velocities}

When changing temperature abruptly in the simulation script, the thermostat will take some time to get the average temperature to the right value. This can be circumvented by scaling the velocities instantaneously:
\begin{verbatim}
# Scale velocities from temperature Told to T
velocity_factor = ( T/Told )**0.5
sim.sample.ScaleVelocities( velocity_factor )
sim.itg.SetTargetTemperature( T )
\end{verbatim}

\subsection{Starting a simulation from where another left off}

It can be as simple
as taking the file, say ``end.xyz.gz'', that you wrote at the end of the last
simulation and using this as the start file for the next. If you want perfect
continuity, you might want the integrator meta-data to be used. This will only
happen if the file is read after the integrator has been set, so you need to 
re-read the start file:

\begin{verbatim}
sim = Simulation("start.xyz.gz")
...
itg = ...
...
sim.SetIntegrator(itg)
sim.sample.ReadConf("start.xyz.gz") 
\end{verbatim}




\section{Simulation methods}

%% Note \Q gives a verbatim that can be used as a description key
%% which is necessary to handle underscores in variable names
%% We use it on all keys here to get a consistent font

To get a presentation of the internal documentation use
\verb|help(Simulation)| within Python or \verb|pydoc Simulation| on the
command line. The docstrings for \verb|Simulation| methods intended
to be called by the user are reproduced here for convenience.

\begin{description}
\item[\Q{AddExternalCalculator(self, calc)}]
  Add data computed by an external calculator class to the energies files.
  
  Example
\begin{verbatim}
    alt_pot_calc = AlternatePotentialCalculator(...)
    sim.AddExternalCalculator(alt_pot_calc)
\end{verbatim}
  

\item[\Q{AddOutputManager(self, manager_name, manager_obj)}]
  Add an existing output manager object (typically a C++ output
  manager), specifying a name which will be used to refer to the
  object when calling SetOutputScheduling and SetOutputMetaData.
  The name can be anything, but the convention is that it matches
  the names of the output files associated with this manager.

\item[\Q{AddPotential(self, potential, clear=False)}]
  Add a potential for this system
  (the total force vector and potential will be the sum over all
  added potentials). Set \verb|clear=True| to remove previously added potentials.
  

\item[\Q{GetNumberOfParticles(self)}]
  Return the number of particles in the sample object.

\item[\Q{GetVolume(self)}]
  Return the volume defined by the simulation box.


\item[\Q{NewOutputManager(self, name)}]
  Create a new (Python) output manager, which will write files in the
  output directory starting with name.
  
\item[\Q{ReadMoleculeData(self, topologyFile)}]
  Read data describing how particles are connected as molecules
  and create the MoleculeData object.
  

\item[\Q{RegisterCallback(self, manager, function, **kwargs)}]
  Register a data analysis function (which takes a Sample as argument)
  to be called with a specified time-step interval. Callback functions
  that are used to generate output should return a string without newline
  characters.

\item[\Q{}RemoveExternalCalculator(self, calc)] Remove/disassociate the calculator object \verb|calc| from the energies output manager
  
\item[\Q{RemoveRuntimeAction(self, name)}]
  Remove an item from the list of runtime-actions

\item[\Q{ResetMomentum(self)}]
  Sets the total momentum to zero via a Galilean velocity transformation

\item[\Q{Run(self, num_iter, initializeOutput=True,restartBlock=None, suppressAllOutput=False)}]
  Run \verb|num_iter| time steps, by default initializing/resetting output.
  Specify \verb|restartBlock| to restart a simulation which has been
  interrupted. Set \verb|suppressAllOutput| to \verb|True| to avoid any output
  (for example for equilibration). Set \verb|initializeOutput| to \verb|False|
  to continue the last run without re-initializing all of the output managers.

\item[\Q{ScaleSystem(self, scaleFactor, direction=None, CM=False)}]
  Scale the simulation box and particle positions by scaleFactor
  in all directions, if direction is omitted; otherwise only
  in the direction direction. For molecular systems, the centers of 
  mass can be scaled while keeping intra molecular distances constant.


\item[\Q{SetBlockSize(self, blockSize)}]
  Specify the block-size for output (None means automatic). Should be a
  power of 2 for logarithmic scheduling.
  

\item[\Q{SetIntegrator(self, itg)}]
  Set the integrator for this system, and call its momentum-resetting
  function.
  
\item[\Q{SetMomentumResetInterval(self, mom_reset_interval)}]
  Set how many time steps should go between resetting of
  center of mass momentum to zero.
  
\item[\Q{SetOutputDirectory(self, outputDirectory)}]
  Change the directory for output files, default TrajectoryFiles.
  
\item[\Q{SetOutputMetaData(self, manager, **kwargs)}]
  Access to the output manager to control precision and what gets written.
  
  Examples
  
\begin{verbatim}
  sim.SetOutputMetaData("trajectory", precision=6, virials=True)
  sim.SetOutputMetaData("energies", potentialEnergy=False)
\end{verbatim}  

\item[\Q{SetOutputScheduling(self, manager, schedule, **kwargs)}]
  Set scheduling information for an output manager.
  
  manager must be one of the current managers, which include "energies"
  and "trajectory" and whatever other managers have been added.
  
  schedule must be one of "none", "linear","logarithmic","loglin"
  
  extra keyword arguments may/must be supplied where relevant, e.g.
  interval=100 for linear scheduling (required).
  base=10 for logarithmic scheduling (optional, default base is 1)
  base=1, maxInterval=16 for loglin (required)
  
  
  
\item[\Q{SetRuntimeAction(self, name, method, interval)}]
  Specify that a 
  user-supplied bound method, taking no arguments, should be called during the 
  main loop (before the force calculation) every \verb|interval| time steps.
  \verb|name| is a string used to refer to this action subsequently.

\item[\Q{SetSimulationBox(self, simBox)}]
  Set an alternative simulationBox.

%\item[\Q{SetSortInterval(self, sort_interval)}]
%  Set how many time steps should go between sorting of
%  particle positions (default value 200). Not necessary to be
%  called by user if using autotune.

\item[\Q{SetVerbose(self, vb)}]
  Turn on/off most messages in \verb|Simulation| and sample objects.
  sim.SetVerbose automatically calls sim.sample.SetVerbose; call
  to the latter explicitly to only affect sample messages.
  
\item[\Q{WriteConf(self, filename)}]
  Write the current configuration to a file.
    
\item[\Q{WritePotentials(self)}]
  Write a file containing the pair potential
  and pair force for each associated pair potential.
  
\item[\Q{__init__(self, filename, pb=32, tp=4, verbose=True}]
  Create the simulation object, and the underlying sample object, after
  reading in a configuration from the named file. Assign default values
  to various simulation- and output-related parameters. Parameters pb
  and tp set the number of particles per block and threads per particle
  respectively. verbose=False suppresses messages to standard output.

\end{description}


\section{\label{Sample_methods}Sample methods}

Here is a list of sample methods which may occasionally be useful in a user-script. Methods which correspond to ones in \verb|Simulation| (which simply call the \verb|Sample| ones) are not listed.

\begin{description}
% Basic data about the system (NumberOfParticles is a method of Simulation)
\item[\Q{GetNumberOfTypes()}]
  Return the number of particle types present.
\item[\Q{GetMass(type)}]
  Return the mass associated with a given type.
\item[\Q{GetMeanMass()}]
  Return the concentration-weight average mass.
\item[\Q{GetNumberOfDOFs()}]
  Return the number of degrees of freedom.
\item[\Q{SetNumberOfDOFs(DOFs)}]
  Set the number of degrees of freedom manually. This is normally done automatically, but manual setting may be necessary in systems with a lot of overlapping constraints.
% dynamic data: positions etc. Useful for on-the-fly analysis.
\item[\Q{GetPositions()}]
  Return the positions as a numpy array.
\item[\Q{GetVelocities()}]
  Return the velocities as a numpy array.
\item[\Q{GetForces()}]
  Return the forces as a numpy array.
\item[\Q{GetImages()}]
  Return the images as a numpy array.
\item[\Q{CalcF(calc_stresses=False)}]
  Carry out force calculation to make sure the forces and energies are up to date. Include \verb|calc_stresses=True| if stress(es) are required.
\item[\Q{GetPotentialEnergy()}]
  Return the total potential energy. Assumes \verb|CalcF| has been called since positions were last changed.
\item[\Q{GetVirial()}]
  Return the total virial. Assumes \verb|CalcF| has been called since positions were last changed.
\item[\Q{GetStress()}]
  Return the system's (atomic) stress as a numpy array of length 6. Assumes \verb|CalcF| has been called since positions were last changed.
\item[\Q{SetIncludeKineticStress(include_kinetic_stress)}]
  Determine whether the (atomic) stress includes the kinetic contribution.
\item[\Q{IsotropicScaleSystem(Rscal)}]
  Scale isotropically by linear factor \verb|Rscal|
\item[\Q{AnisotropicScaleSystem(Rscal, dir)}]
  Scale along the coordinate direction given by \verb|dir| by the amount \verb|Rscal|
\item[\Q{IsotropicScaleSystemCM(Rscal)}]
  Scale a molecular system isotropically by linear factor \verb|Rscal|. The molecules' centers of mass will be rescaled, but the size of molecules is not changed.
\item[\Q{AffinelyShearSystem(shear_strain)}]
  Make a homogeneous simple shear by the given amount of strain.
\item[\Q{ScaleVelocities(factor)}]
  Scale all velocities by a given factor.
% access to 
\item[\Q{GetSimulationBox()}]
  Return the \verb|SimulationBox| object. For example if you want to replace it with a \verb|LeesEdwardsSimulationBox|, you pass it as a constructor argument to the latter.

% Functions related to more detailed control of input/output behavior.
\item[\Q{GetOutputDirectory()}]
  Get name of the output directory (TrajectoryFiles by default)
\item[\Q{EnableBackup(make_backup)}]
  Enable/disable making of back-up output directories.
\item[\Q{SetLogLinParameters(manager_name, base, maxInterval)}]
  Specify log-lin parameters for an output manager beyond the standard options of linear (corresponding to base = maxInterval = interval) and logarithmic (corresponding to maxInterval=0, with base=1 by default, although other values can be passed).
\item[\Q{SetVerbose(vb)}]
  Change verbosity for sample, independently of the Simulation object.
\item[\Q{TerminateOutputManagers()}]
  Make sure all output files are closed. Necessary if doing analysis is to be done in the same python script (or session) as the simulation. 
\item[\Q{ReadConf(filename)}]
  Read a new configuration into the sample from the given file.
% CUDA related control
\item[\Q{SetCheckCudaErrors(set_check_cuda_errors)}]
  Check for CUDA errors once every time step for debugging purposes.
\item[\Q{GetParticlesPerBlock()}]
  Return $p_b$
\item[\Q{GetThreadsPerParticle()}]
  Return $t_p$
\end{description}


\section{Some internal technical details of possible 
  relevance to users}

\subsection{Copying}

Copying of simulation objects is important for the autotuner, which needs to
re-run the same simulation many times and leave the simulation object in its
original state.

When copying a sample object, it is assumed the data on the device is the most
up to date. Therefore a call to CopyDataFromDevice is made on the object being 
copied; then the host data is copied over to the new object.



\end{document}

