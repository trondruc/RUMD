%\documentclass{article}
%\usepackage{verbatim}
%\usepackage{graphicx}
%\usepackage{amssymb,amsmath}

\section{Level 2: Fine control} 

%\author{LB}
%\date{Level 2}
%\begin{document}
%\maketitle


The user at level 2 can use the full \verb|Simulation| interface
to control many details of the simulation. Here we 
present, mostly 
in the form of python code-snippets, some of the more advanced 
aspects of the interface, though
without trying to present a complete documentation. The next
step is to look on the RUMD web-page (\verb|rumd.org|) under ``example
scripts'' for ideas about how to get the most out of RUMD.


\subsection{Controlling how the simulation run is divided into blocks}

The simulation output is divided into blocks of size \verb|blockSize|. 
By default the block-size is chosen when \verb|Run| is called, so that
there will be of order 100-200 blocks in the given number of time-steps.
Sometimes it is useful to explicitly specify the block-size (this will sometimes be necessary when using log-lin because there are some constraints on the 
block-size in relation to the log-lin parameters). The following will set
the blockSize (we denote by \verb|sim| the \verb|Simulation| object):

\begin{verbatim}
  sim.SetBlockSize(1024)
\end{verbatim}

\subsection{Controlling what gets written to energies and trajectory files}

To change what gets written we call the \verb|SetOutputMetaData| method, which
functions similar to \verb|SetOutputScheduling| in that it takes a string with 
the name of the output manager, and one or more keyword arguments specifying 
either that a parameter should be included or not, or what the overall 
precision should be.


\begin{verbatim}
  sim.SetOutputMetaData("energies",totalEnergy=False,stress_xy=True, precision=7)
\end{verbatim}
which turns off writing the total energy, turns on writing the xy component
of the stress tensor, and sets the precision to seven decimal places. 
Table~\ref{energyQuantities} shows the most important available
quantities. To each of these are associated two strings: an ``identifier'' for
use in scripts, to refer to a given quantity when turning it ``on'' or ``off'';
and a ``column-label'', which appears in the comment-line of the energies file
to identify which column corresponds to which quantity.

\begin{table}
  \caption{\label{energyQuantities}Identifier and column-label for the main
    quantities that can be written to the energies file.}
  \begin{center}
    \begin{threeparttable}
      \begin{tabular}{|c|c|}
        \hline
        identifier & column label \\
        \hline
        \verb|potentialEnergy| & \verb|pe| \\
        \verb|kineticEnergy| & \verb|ke| \\
        \verb|virial| & \verb|W| \\
        \verb|totalEnergy| & \verb|Etot| \\
        \verb|temperature| & \verb|T| \\
        \verb|pressure| & \verb|p| \\
        \verb|volume| & \verb|V| \\
        \verb|stress_xx| & \verb|sxx| \tnote{(a)} \\
        \hline
      \end{tabular}
      \begin{tablenotes}
      \item[(a)] Potential part of atomic stress. Similarly yy, zz, xy, etc
      \end{tablenotes}
  \end{threeparttable}
  
\end{center}
\end{table}



For simulations in which more than one potential is present, 
the contributions from different terms
to the total potential energy can be included in the energies file. Each 
potential class has an ID string. For example for \verb|Pot_LJ_12_6| it is 
``potLJ''. If desired the ID string can be changed (for example to the simpler 
``LJ'') via

\begin{verbatim}
  pot.SetID_String("LJ")
\end{verbatim}
In the case of contributions to potential energy this ID-string functions as
both the identifier in scripts, and the column-label in the energies-file.
So for example, after changing the ID-string to ``LJ'' as above, to cause it 
to be written we would write

\begin{verbatim}
  sim.SetOutputMetaData("energies", LJ=True)
\end{verbatim}
There is a mechanism for additional quantities to be defined and included in
the energies file, via classes of type \verb|ExternalCalculator|. This is
described in the user manual.

\subsection{Changing the sample volume within the script}

It can be useful to take a configuration corresponding to one density and
rescale the box and particle coordinates such that it has a different density.
This is accomplished by the following lines (which also show how to get the
current volume and number of particles).

\begin{verbatim}
  nParticles = sim.GetNumberOfParticles()
  vol = sim.GetVolume()
  currentDensity = nParticles/vol
  sim.ScaleSystem( pow(newDensity/currentDensity, -1./3) )
\end{verbatim}
where \verb|newDensity| is understood to be the desired density.

\subsection{Options for Run: Restarting a simulation, 
suppressing output, continuing output}

Under default conditions a ``restart file'' will be written, once per 
output-block. If a simulation is interrupted (due to hardware problems for 
example), it can be restarted from the last completed 
block (here 12) by
 
\begin{verbatim}
  sim.Run(20000, restartBlock=12)
\end{verbatim}
where the original number of time-steps should be specified, and the last
completed block can be found by looking at the file 
``LastComplete\_restart.txt'' in TrajectoryFiles. It is possible to suppress 
writing of restart files (so restarts will not be possible), 
and all other output, by calling

\begin{verbatim}
  sim.Run(20000, suppressAllOutput=True)
\end{verbatim}
This might be done for example, during an initial period of equilibration, 
although if this is planned to take a long time, it may be advisable to allow
restart files to be written, in case of possible interruptions.

To avoid re-initializing the output, so that repeated calls to Run will 
append to the existing energies and trajectory files, do

\begin{verbatim}
  sim.Run(20000, initializeOutput=False)
\end{verbatim}
This will give an error if \verb|Run| was not called previously in the normal
way (\verb|initializeOutput=True|, which is its default value).


\subsection{Getting optimum performance via the autotuner}

RUMD involves several internal parameters whose values do not affect the 
results of the simulation within round-off
error, but they can have a noticeable effect on performance. The easiest way
to identify the optimal values is to use the autotuner, which is a python
script that takes a simulation object which is ready to run (the potential(s)
and integrator have been set), and determines the best values for the internal
parameters. The following snippet shows how this is done.

\begin{verbatim}
from rumd.Autotune import Autotune
at = Autotune()
[create sim with associated potential(s) and integrator]
at.Tune(sim)
sim.Run()
\end{verbatim}

The time taken to do the tuning is of order a minute for small systems
(longer if constraints are involved). The autotuner writes a file,
\verb|autotune.dat| in the directory. If the simulation is re-run in the same
directory with the same user parameters and on the same type of graphics card, 
then the autotuner will simply use the previously determined internal 
parameters.

\clearpage

%\end{document}
