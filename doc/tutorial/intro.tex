%\documentclass{article}

%\usepackage{geometry}

%\begin{document}

\section{Introduction}
The CUDA project has enabled scientists to utilise NVIDIA's many-core
Graphical Processor Unit (GPU) to solve complex computational
problems. Beside being a very fast processor the GPU is relatively
cheap, and is thus an ideal computational unit for low-cost high
performance supercomputing. Scientific software is currently being
developed in a very rapid pace by many different groups world wide,
and today there exists optimized software for fluid dynamics and
matrix operations just to name a few.   

RUMD [`r\textopeno m di\normalsize$^\bullet$]  
is a high-performance molecular dynamics simulation software package
optimized for NVIDIA's graphics cards. RUMD is designed for small to
medium size systems of spherical particles and simple molecules. RUMD
is developed at The Danish National Research Foundation's centre 
``Glass and Time'' which is located at the Department of Sciences,
Roskilde University.   

\subsection{Features}
RUMD currently supports: 
\begin{itemize}
\item van der Waals type pair potentials: Lennard-Jones, Gaussian
  core, inverse power law, exponential, and more. It is easy to implement new pair
  potentials. 
\item Multicomponent simulations
\item Bond stretching potentials: Harmonic and FENE
\item Bond constraint dynamics
\item Angle bending potentials
\item Dihedral  potentials
\item NVE, NVT, NPT (atomic) and NVU  ensemble simulations
\item Non-equilibrium sheared simulations using Lees-Edwards boundary conditions and the SLLOD equations of motion
\item A Python interface including user-defined on-the-fly analysis.
\item Tools for setting up configuration files, including molecular systems.
\item Post simulation analysis tools
\end{itemize}
In the near future we plan to implement:
\begin{itemize}
\item NPT (molecular) ensemble 
\item Ewald-based Couloumb interactions (a Coulomb potential using direct pair summation and a shifted-force cut-off is available which gives reasonable performance and accuracy for bulk ionic systems)
\end{itemize}

\subsection{User philosophy}
We have defined five users of RUMD:
\begin{enumerate}
\item A user at level 1 is one who can write or copy a simple python script
(for example the one given later in this tutorial) to run a standard molecular 
dynamics simulation, and use the provided analysis tools to perform standard
analysis of the output. This user can make some basic choices, such as the
potential, the integrator, and controlling the frequency of output.
\item This user understands the deeper structure of the code---that is
the different kinds of classes/objects, the relations between them, and their 
interfaces---and understands how the python language works, and can therefore 
write more sophisticated python scripts which
access the full power and flexibility of RUMD.
\item A user at level 3 is one with minimal experience in C++ can write
her own (pair-)potential functions, to be available as classes derived from
\verb|PairPotential|.
\item This user can write their own analysis programs in C++, but does not
need to know anything about GPU programming, since analysis is generally
carried out on the CPU.
\item RUMD developer. Must have a good knowledge of C++, CUDA programming
and knowledge of Subversion.
\end{enumerate}
This tutorial will bring you from level 1 to level 3. Note that level 3 is
higher than level 2 in that it requires the user to write some C++ code and
re-compile the code, but that involves a relatively limited list of things to 
do. Level 2, which stays at the python level, but involves access to almost all of 
the underlying C++ functionality, could take longer to master.


\subsection{Installation}
For Debian-like systems, RUMD can be installed as a package. 
Use the following repository entries appended to 
\textsf{/etc/apt/sources.list} 
or in a file in \textsf{/etc/apt/sources.d/} 
\begin{verbatim}
deb [trusted=yes] http://carid.ruc.dk/rumd buster main
deb-src [trusted=yes] http://carid.ruc.dk/rumd buster main
\end{verbatim}
To install RUMD simply use an equivalent of:
\begin{verbatim}
apt update
apt install rumd rumd-doc python3-rumd
\end{verbatim}
You need to have root privileges, of course. 

You can also build the package from the source. To do so, download the
source code from the RUMD homepage 
\begin{verbatim}
http://rumd.org/download.html
\end{verbatim}
and unpack the file using
\begin{verbatim}
tar xzvf rumd-X.X.tar.xz
\end{verbatim}
where \verb|X.X| is the version number. Enter the source directory 
\begin{verbatim}
cd rumd-X.X
\end{verbatim}
and build the package by typing 
\begin{verbatim}
make
\end{verbatim}
To test your installation use
\begin{verbatim}
make test
\end{verbatim}
This test will take approximately 5-10 minutes.

Notice, you cannot type \verb=make install= since the default path is 
different from system to system. We will refer to the installation
path as \verb=<RUMD-HOME>=  


\subsection{Setting the path for python}

In a proper installation the python  modules should be located in a standard
place where python will automatically find them. This may not be possible if
you do not have root privileges, however. If you have compiled the software
yourself, you need to set the environment variable PYTHONPATH as follows 
For C-shell it would go in your \verb|.cshrc| file; 

\noindent \verb|setenv PYTHONPATH <RUMD-HOME>/Swig:<RUMD-HOME>/Python:<RUMD-HOME>/Tools|

For bash you would add the following to your \verb|.profile|;

\noindent \verb|export PYTHONPATH=<RUMD-HOME>/Swig:<RUMD-HOME>/Python:<RUMD-HOME>/Tools|

\clearpage

%\end{document}
