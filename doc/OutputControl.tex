\documentclass[a4paper]{article}

\setlength{\textwidth}{16 cm}
\setlength{\oddsidemargin}{-1 cm}

\usepackage{verbatim}

\title{Controlling data output in RUMD (C++ interface)}
\author{Nicholas Bailey}


\begin{document}

\maketitle

\section{For the new/basic user}



\subsection{Basic output structure}

\begin{itemize}
\item A simulation run is divided into ``blocks'', whose size (number of time 
steps) is typically chosen so that there are of order 100-1000 blocks.
\item The Sample object maintains a list of ``output managers'', each of which
specializes in a certain kind of output and writes it to corresponding files
 (one per block, but the user does not need to be aware of the details of the 
files themselves).
\item By default there are three output managers: one for writing ``restart 
configurations'' (in order to restart the simulation in the event of a crash or 
other failure), one for writing the ``trajectory''---configurations to be used 
for analysis, and one for writing ``energies''---various global quantities such
 as potential and kinetic energies, pressure, etc.
\item The output managers have the facility to write equally spaced in time
 (``linear saving'') or equally spaced in log-time (``logarithmic saving'') as
well as something in between which will be explained in a later section.
\end{itemize}

\subsection{Output methods on sample}


The following class methods (member functions) of the sample class are the
most relevant. Since this section is for a user who is relatively new to
C++, we give sample code as would appear in the main function, rather than the
C++ declarations including type information:

\begin{verbatim}
S->SetOutputBlockSize(blockSize); // set the block size for all output managers.
S->SetWriteTrajectory(true); // (or false) turn on/off writing of trajectory (true by default)
S->SetWriteEnergies(true); // (or false) turn of/off writing of energies (true by default)
S->SetTrajectoryLinearWriteInterval(writeInterval); // pure "linear saving" of trajectory
S->SetTrajectoryLogLinParams(base, 0); // pure "logarithmic saving" of trajectory
S->SetEnergyLinearWriteInterval(writeInterval); // pure linear saving of energies
S->SetEnergyLogLinParams(base, 0); // pure logarithmic saving of energies
S->NotifyOutputManagers(); // called during the main simulation loop
S->SetVerbose(true);// (or false) whether to write print statements to standard
// output (true by default); sets corresponding variables in potential(s) and output managers 
S->WriteConf("last_config.xyz.gz"); // write the current configuration to a file
\end{verbatim}


\noindent Note that restart configurations cannot be turned off. Note also that
there the block size is common to all output managers. The name of the directory
where the files are to be located is specified by a private variable of Sample.
It is currently hard-coded to be ``TrajectoryFiles'' but some user-control of 
this could be added in the future.

\section{For the more advanced user/developer}

\subsection{Class overview}

Some more background information about how functionality is divided among
different classes is necessary to understand the more advanced parts of the
interface.


\begin{description}
\item [LogLinOutputManager] Provides the basic functionality of managing 
linear/logarithmic/log-lin saving. Log-lin saving is a kind of interpolation 
between pure linear and pure logarithmic which includes these two as special
cases. This is an abstract base class; derived classes must 
implement a \verb|Write()| function.
\item [ConfigurationsLogLin] Derives from \verb|LogLinOutputManager|, used for
the restart configurations and the trajectory configurations (via different 
objects of this type). Relatively
light-weight, since it passes the actual job of writing to a
 \verb|RUMD_ConfWriterReader| object.
\item [EnergiesLogLin] Derives from \verb|LogLinOutputManager|, used for 
writing the
energies. Details of what exactly is to be written are managed by an object
of type \verb|RUMD_EnergiesMetaData|.
\item [RUMD\_ConfWriterReader] Does writing (and reading) of configurations in 
the RUMD format. Details of exactly what items are to be included in 
configuration files are managed by an object of type \verb|RUMD_ConfMetaData|.
\item [RUMD\_ConfMetaData] Manages meta-data for configuration files: the
 precision, the values of meta-data to be written in the comment line, and 
information about which data-columns are to be included.
\item [RUMD\_EnergiesMetaData] Manages meta-data for energy files: the 
precision, the values to be written in the comment line, and what data-columns
are to be included.
\end{description}


Note that \verb|EnergiesLogLin| does the writing itself, while
separate classes for writing (and reading) exist only for configurations. It
is important that they exist independently of the log-lin scheduling, because
the user should be able to decide to write a single configuration (e.g. call
\verb|S->WriteConf(filename);|
at the end of the program) independent of whether trajectories were written.


\subsection{Methods and sample code}

The following output-related methods on Sample give more control to the user
(now we give the full method declarations).

\begin{verbatim}
class Sample {
public:
  RUMD_ConfMetaData& GetTrajectoryConfMetaData();
  RUMD_EnergiesMetaData& GetEnergiesMetaData();
  void AddOutputManager(LogLinOutputManager* om); // add a new output manager 
     // beyond the three present by default.
  void InitializeOutputManagers(unsigned long int timeStepIndex); // does what it
     // says. This will be called by NotifyOutputManagers if necessary, but 
     // sometimes  it might be useful for the user to control when to call it
  void TerminateOutputManagers(); // closes open files etc. Not necessary if doing
     // a single run
};
\end{verbatim}

\verb|RUMD_ConfMetaData| and \verb|RUMD_EnergiesMetaData| contain several 
public members which the user can set directly. For the latter, a function \verb|Set(std::string, bool)| is used to set whether a given quantity should be 
written to the files. The full class declarations are given in the 
appendix, where the reader can see the complete set of options.
Here are some examples of how these might be used in a main function:

\begin{verbatim}
   S->GetTrajectoryConfMetaData().precision = 9;
   S->GetTrajectoryConfMetaData().has_pot_energies = true;
   S->GetTrajectoryConfMetaData().has_virials = true;
   S->GetEnergiesMetaData().precision = 9;
   S->GetEnergiesMetaData().Set("virial", true);
\end{verbatim}

\subsection{Full log-lin  functionality}

The ``log-lin'' technique allows saving either logarithmically
(that is, equally spaced $\log(t_i)$ values), linearly (equally spaced $t_i$
values) or a kind of interpolation of the above. 
 To explain in more detail we will make reference to the
actual data structure used to implement it. The basic structure is a class
called LogLin (not completely shown):

\begin{verbatim}
class LogLin{
 public:
  unsigned long int block;          // identifies the current block
  unsigned long int nextTimeStep;   // identifies the next time step (within the block) to write
  unsigned long int base;           // size of smallest interval between writes (set by the user, fixed)
  unsigned long int index;          // labels items within a block
  unsigned long int maxIndex;       // value of index for the last item in block
  unsigned long int maxInterval;    // overrides log-indexing to give linear (set by the user, fixed)
};
\end{verbatim}


The elements here identify both the parameters of a particular log-lin saving
scheme and the details of the next step to be saved. They are typically 
included in the comment line for a configuration in a trajectory file,
so that analysis programs can identify a given item  (configuration or 
line of energies) and the saving scheme which generated it. A description of
each one follows.

\begin{description}
\item[block] Identifies which output block we are in. Will therefore be the same
for all items written in a given block.
\item[nextTimeStep] Identifies the time step within a block, so starts at zero
in each block and, if using pure logarithmic saving with base equal to one and
maxInterval equal to zero, will then proceed
through 1, 2, 4, 8, \ldots
\item[base] This is the smallest number of time steps considered, and is equal
to 1 in the simple version above. It does not change during a simulation run,
but is a parameter controlled by the user. All relevant intervals are then
multiples of base.
\item[index] Labels items within a block, in that it starts at zero and 
increments by 1 for each item written. As long as maxInterval is set to zero
(pure logarithmic saving), there is a simple relation between index and 
nextTimeStep: nextTimeStep = 0 if index = 0 and nextTimeStep = 
base $\times$ $2^{\textrm{index-1}}$ otherwise. Note that the size of a block, 
blockSize, is not included explicitly in the LogLin class, but is given
(again for pure logarithmic saving, maxInterval=0) by 
base $\times 2^{\textrm{maxIndex-1}}$ 
\item[maxIndex] The value of ``index'' for the last item in the block. So if
the saving scheme is such that there are 8 items per block, maxIndex will have
the value 7 (since index starts at zero). Does not change during a simulation
run\footnote{The exception to this is linear saving where the interval does not evenly divide the blockSize.}; is set based on parameters specified by the user.
\item[maxInterval] The largest interval allowed between saves. This is a user
parameter which does not change. Its purpose is to allowed combined linear and
logarithmic saving. It should be set to zero to do pure logarithmic saving.
If non-zero, the interval between writes will, after increasing initially
logarithmically become fixed at maxInterval. There is a more complicated 
relation between nextTimeStep and index in this case.
\end{description}

Some possible combinations are, assuming blockSize=128

\begin{description}
\item[Pure logarithmic saving with base 1] Set base=1, maxInterval=0, and you
get time steps 0, 1, 2, 4, 8, 16, 32, 64, 128
\item[Pure linear saving at interval 4 time steps] Set base=4 and maxInterval=4
and you get time steps 0, 4, 8, 12, 16, \ldots, 128
\item[``Interpolated'' log-lin] Set base=1 and maxInterval=8, and you
get time steps 0, 1, 2, 4, 8, 16, 24, 32, 40, 48, 56, \ldots, 128
\end{description}


Note the following restrictions on the values of base
and maxInterval:  When maxInterval is non-zero it must be a 
multiple of base. Except for the case of linear saving (base=maxInterval), both
base and maxInterval must divide evenly into blockSize. 



\subsection{Motivation and discussion}

This motivation section has been placed at the end of document so that people 
who need to know the basics of the interface can quickly find what they need
at the start. The intention here is to explain the necessity/usefulness of
 having a relative complex system for controlling data output, as well as the
reasons for some of the design decisions.


The simulations we are most interested in run for a long time 
($10^8$-$10^{10}$ time steps), and we are interested in dynamics on both short 
and long time scales. Because of the presence of long relaxation times, even if
interested in only short times a short simulation is not useful 
because it is necessary to average over long times.
Writing data (configurations as well as thermodynamic
data such as energies and pressures, for simplicity referred to collectively
as ``energies'' ) at equally-spaced intervals is not appropriate because the 
requirement of short-time dynamical information means that the write-interval
should be short, but then because the simulations are so long we end up writing
vast amounts of data and filling up disk space too rapidly.

A possible method for sampling short-time  dynamics is the following, which 
could be characterized as ``intermittent short-time saving'': We 
divide the simulation time
into $n_b$ ``output blocks'' where $n_b$ is expected to be of order 100-1000. 
Therefore each block a size $l_{block}$ of order
 $10^5$-$10^8$ time steps. We  write 
data for only the first 1000 (or whatever) time steps of each block. It would be
perhaps one percent of the total. This gives
sufficient data to resolve dynamics on time scales up to 1000 time steps, which
can be properly averaged over long times. By considering the first time step
in each block dynamical information at the longest time scales is also 
available.

The limitation of the above approach is that intermediate time scales are
not available. To have a broad range of time scales we need to save 
``logarithmically'' within blocks. In the simplest version we assume the block 
size is a power of 2. Labelling the time steps within a block by indices 
0, 1, 2, 3, 4, \ldots,  we write data at time indices $t_i$=0, 1, 2, 4, 8, 16,
\ldots, $l_{block}$; that is, zero and all powers of two up to the block size. 
Note that in this scheme data for the time step $t_i=l_{block}$ gets written
twice, both as the end of a block, and as the start of the following block.
This redundancy has negligible cost and can be helpful for ensuring consistency
and simplifying analysis. In this scheme the number of time steps for which 
data is saved is $n_b (2+\log_2(l_{block}))$ which is a tiny fraction of the 
total, assuming $l_{block} \gg 1000$.
Considering time {\em differences} (which are relevant to analysis of dynamics)
there are more available than just powers of two,  in fact 
$\Delta t_i$=1, 2, 3, 4, 6, 8, 12, 14, 16, \ldots, although analysis programs
may choose not to use all possible differences. 


The function \verb|AddOutputManager| allows the user to add new objects
to the list of log-lin managers, for example to be able to add more than one 
kind of scheduling to
be active in a given simulation concurrently. OutputManger objects provided
from outside by the user are assumed to be owned by the user---Sample does not
destroy them, while it does destroy the three default ones in its destructor.


\subsection{Possible future extensions}

With the existing class structures it becomes easy to generalize:

\begin{itemize} 
\item Making classes that write (and read---it makes sense to keep code for
writing and reading a given format together in the same class) configurations 
in alternative formats. Technically this means defining an (abstract) base class
\verb|ConfWriterReader| from which other writer classes could be derived. 
It/they would have the same interface as \verb|RUMD_ConfWriterReader|, and
one would pass a base-class pointer to \verb|ConfigurationsLogLin|
\item Generalizing from
log-lin to other kinds of scheduling, for example
the intermittent short-time saving protocol described above, 
which could still be useful in cases where one was interested in 
having very detailed short-time dynamical information averaged over long
times. In this case there would be a new base class \verb|OutputManager|
from which \verb|LogLinOutputManager| would derive, along with other, non
log-lin based, output scheduling classes.
\end{itemize}

\appendix

\section{Meta-data class declarations}

Here we present the definitions of the two meta-data classes, so
that the list of available options can be seen. The interfaces are
slightly different, in particular \verb|RUMD_EnergiesMetaData|uses
Set and Get functions to set the attributes which specifies whether a given
quantity should be written or not. Non-public parts of the class are not shown.

\subsection{RUMD\_ConfMetaData}




\begin{verbatim}
class RUMD_ConfMetaData
{
public:
  RUMD_ConfMetaData():
    precision(6),
    rumd_conf_ioformat(0), // refers to a file, so could be different from the \
current rumd_conf_ioformat                                                      
    numTypes(1), // refers to a file                                            
    timeStepIndex(0),
    numMolecules(0),
    L(),
    Ps(0.),
    Pv(0.),
    dt(0.),
    massOfType(),
    has_images(false),
    has_velocities(false),
    has_forces(false),
    has_pot_energies(false),
    has_virials(false),
    has_moleculeIndices(false),
    has_logLin(false),
    blockSize(0),
    logLin()
  {
    // might want to remove this and place the burden of initialization         
    // on the calling function                                                  
    for(unsigned int cd = 0; cd < DIM; ++cd)
      L[cd] = 0.;
    for(unsigned int tdx = 0;tdx < MaxNumTypes; ++tdx)
      massOfType[tdx] = 1.;
  }
  // the function which parses the comment line for configuration files         
  void ReadMetaData(char *commentLine, bool verbose);

 unsigned int precision;
  // might want to add a separate precision for energies/virials                
  unsigned int rumd_conf_ioformat;
  unsigned int numTypes;
  unsigned int timeStepIndex;
  unsigned int numMolecules;
  float L[DIM];
  float Ps;
  float Pv;
  float dt;
  float massOfType[MaxNumTypes];
  bool has_images;
  bool has_velocities;
  bool has_forces;
  bool has_pot_energies;
  bool has_virials;
  bool has_moleculeIndices;
  bool has_logLin;
  unsigned int blockSize;
  LogLin logLin;
\end{verbatim}

\subsection{RUMD\_EnergiesMetaData}



\begin{verbatim}
class RUMD_EnergiesMetaData
{
public:
  RUMD_EnergiesMetaData():
    rumd_energies_ioformat(0),
    precision(6),
    writeInterval(100),
    usingAtomicVirPress(true),
    has_logLin(false),
    timeStepIndex(0),
    logLin(),
    options(),
    fileStr()
  {
    options["potentialEnergy"] = true;
    fileStr["potentialEnergy"] = "pe";
    options["kineticEnergy"] = true;
    fileStr["kineticEnergy"] = "ke";
    options["virial"] = true;
    fileStr["virial"] = "W";
    options["totalEnergy"] = true;
    fileStr["totalEnergy"] = "Etot";
    options["temperature"] = true;
    fileStr["temperature"] = "T";
    options["pressure"] = true;
    fileStr["pressure"] = "p";
    options["volume"] = false;
    fileStr["volume"] = "V";
    options["thermostat_Ps"] = true;
    fileStr["thermostat_Ps"] = "Ps";
    options["enthalpy"] = false;
    fileStr["enthalpy"] = "H";
    options["potentialVirial"] = false;
    fileStr["potentialVirial"] = "pot_W";
    options["constraintVirial"] = false;
    fileStr["constraintVirial"] = "con_W";
    options["stress_xx"] = false;
    fileStr["stress_xx"] = "sxx";
    options["stress_yy"] = false;
    fileStr["stress_yy"] = "syy";
    options["stress_zz"] = false;
    fileStr["stress_zz"] = "szz";
    options["stress_yz"] = false;
    fileStr["stress_yz"] = "syz";
    options["stress_xz"] = false;
    fileStr["stress_xz"] = "sxz";
    options["stress_xy"] = false;
    fileStr["stress_xy"] = "sxy";
    options["v_com_x"] = false;
    fileStr["v_com_x"] = "v_com_x";
    options["v_com_y"] = false;
    fileStr["v_com_y"] = "v_com_y";
    options["v_com_z"] = false;
    fileStr["v_com_z"] = "v_com_z";
    options["configTemperature"] = false;
    fileStr["configTemperature"] = "configT";
    options["laplacian"] = false;
    fileStr["laplacian"] = "laplacian";
    options["simulationDisplacementLength"] = false;
    fileStr["simulationDisplacementLength"] = "dispLength";
    options["instantaneousTimeStepSq"] = false;
    fileStr["instantaneousTimeStepSq"] = "dt^2";
    options["euclideanLength"] = false;
    fileStr["euclideanLength"] = "eclLength";
  }
  void Set(std::string key, bool val);
  bool Get(const std::string& key);
  const std::string& GetFileString(const std::string& key);
  unsigned int rumd_energies_ioformat;
  int precision;
  int writeInterval;
  bool usingAtomicVirPress;
  bool has_logLin;
  unsigned int timeStepIndex;
  LogLin logLin;
  // the function which parses the comment line for configuration files         
  void ReadMetaData(char *commentLine, int verbose);


\end{verbatim}

Note that for each quantity there is a string which labels it internally as 
well as a ``file string'' which appears in the comment line of the file and
labels the corresponding column.

\end{document}
